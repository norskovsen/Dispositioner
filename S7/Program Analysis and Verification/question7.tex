\section{Basic Separation Logic: Specs and Proof for Abstract Data Types}
\subsection{Basic Seperation Logic}
\subsubsection{Star magic wand and weak rule}
\begin{itemize}
  \item A proposition $P$ escribes a set of resources
  \begin{itemize}
  	\item $\mathcal R$ is used for the set of resources and $r_1$, $r_2$ etc. for elements in $\mathcal R$
    \item It is assumed that 
    \begin{itemize}
    	\item There is an empty resource
      \item There is a way to compose resources $r_1$ and $r_2$ denoted $r_1 \cdot r_2$
      \item The composition defindes for resources are disjoint denoted $r_1 \# r_2$
    \end{itemize}
  \item The intuition for $P \vdash Q$ is that all resources in $P$ are also in $Q$ i.e.
  \begin{equation*}
    \forall r \in \mathcal R. r \in P \Rightarrow r \in Q 
  \end{equation*}
  \end{itemize}
  \item $P*Q = \{r \mid \exists r_1, r_2, r= r_1 \cdot r_2 \land r_1 \in P \land r_2 \in Q \}$
  \item $P \magic Q = \{r \mid \forall r_1. r_1 \# r \land r_1 \in P \Rightarrow r \cdot r_1 \in Q\}$
  \item The following rule $*$-\texttt{WEAK}
  \begin{equation*}
    \frac{}{P_{1} * P_{2} \vdash P_{1}}
  \end{equation*}
  makes iris into a affine seperation logic
  \begin{itemize}
  	\item Intuitively this rules implies that propositions are interpreted by upwards closed set of resources
  \begin{itemize}
  	\item $r_{1} \geq r_{2}$ iff $r_{1}=r_{2} \cdot r_{3},$ for some $r_{3}$
    \item Suppose $r_{1} \in P_{1}$ and that $r \geq r_{1} .$ Then there is $r_{2}$ such that $r=r_{1} \cdot r_{2}$
    \item Let $P_{2}$ be $\left\{r_{2}\right\}$
    \item Then $r_{1} \cdot r_{2} \in P_{1} * P_{2}$
    \item By the weakening rule, we then also have that $r=r_{1} \cdot r_{2} \in P_{1}$
    \item Hence $P_{1}$ is upwards closed.
  \end{itemize}
  \end{itemize}
\end{itemize}

\subsubsection{Hoare triples and points to predicates}
\begin{itemize}
	\item To get basic separation logic the basic logic of resources are extended with two new predicates
  \begin{itemize}
    \item Hoare triples are basic predicates about programs
    \item The points-to predicate $x \hookrightarrow v$ is the basic proposition about resources
  \end{itemize}
  \item The hoare triple satisfy the following type rule
  \begin{equation*}
  \frac{\Gamma \vdash P: \text { Prop } \quad \Gamma \vdash e: E x p \quad \Gamma \vdash \Phi: V a l \rightarrow \text { Prop }}{\Gamma \vdash \{P\} e\{\Phi\}: \text { Prop }}
  \end{equation*}
  The intuitive reading of the Hoare triple $\{P\} e\{\Phi\}$ is that
  \begin{itemize}
  	\item If the program $e$ is run in a heap $h$ satifying $P$, then the computation does not get stuck
    \begin{itemize}
    	\item Computation get stuck when one tries to something illegal such as trying to dereference a location that has not been allocated
    \end{itemize}
  	\item If it terminates with a value $v$ and a heap $h'$ then $h'$ satisfies $\Phi(v)$
  	\item $\Phi$ has two purposes
    \begin{itemize}
  		\item It describes the value $v$
  		\item It describes the resources after the execution of the program
    \end{itemize}
  \end{itemize}
  \item The points to predicate satisfy the following type rule
  \begin{equation*}
    \frac{\Gamma \vdash \ell: \text { Val } \quad \Gamma \vdash v: \text { Val }}{\Gamma \vdash \ell \hookrightarrow v: \text { Prop }}
  \end{equation*}
  \begin{itemize}
  	\item It describes a set of heap fragments that map a location 
    \begin{equation*}
      x \hookrightarrow v = \{h \mid x \in \text{dom}(h) \land h(x) = v\}
    \end{equation*}
    \item Given that one has $\ell \hookrightarrow v$, then one have the ownership of $\ell$ and hence can may modify what $\ell$ pointsto, without invalidating invariants of other parts of the program.
    \item The essential properties of the points-to predicate is that they are not duplicable i.e.
    \begin{equation*}
      \ell \hookrightarrow v * \ell \hookrightarrow v^{\prime} \vdash \text { False }
    \end{equation*}
    and that it is a partial function in the sense that
    \begin{equation*}
      \ell \hookrightarrow v \wedge \ell \hookrightarrow v^{\prime} \vdash v=_{v a l} v^{\prime}
    \end{equation*}	
  \end{itemize}
\end{itemize}

\subsubsection{Hoare triples rules}
\begin{itemize}
  \item The frame rule (\texttt{HT-FRAME})
  \begin{equation*}
    \frac{S \vdash\{P\} e\{v . Q\}}{S \vdash\{P * R\} e\{v . Q * R\}}
  \end{equation*}
  expresses if an expression $e$ satisfies a Hoare triple, then it will preserve any resources described by a ``frame'' $R$ disjoint from the resources described by the precondition $P$
  \item The bind rule (\texttt{HT-BIND})
  \[
  \begin{prooftree} 
    \hypo{E\text{ is an eval. context}}
    \hypo{S \vdash \{P\} e \{v.Q\}}
    \hypo{S \vdash \forall v. \{Q\} E[v] \{w.R\}} 
    \infer3{S \vdash \{P\} E[e] \{w.R\}}
  \end{prooftree}
  \]
  is used to transform the verification of a big program $E[e]$ to the verification of individual steps for which there is basic axoims for Hoare triples
  \item \textbf{Persistent propositions} is propositions which do not rely on any exclusive resources
  \begin{itemize}
  	\item The essential properties of persistents propositions are the \texttt{HT-EQ} and \texttt{HT-HT}
  	\item The following axiom holds for any \textbf{persistent propositions} $P$ and any proposition $Q$
    \begin{equation*}
      P \wedge Q \vdash P * Q \quad \text { if } P \text { is persistent. }
    \end{equation*}
    \item Intuitively, if $P$ is permistent, then it does not depend on any exclusive resources
    \item The following entailment always holds
    \begin{equation*}
    P * Q \vdash P \land Q
    \end{equation*}
    \end{itemize}
  \item The rule of consequence (\texttt{HT-CSQ}):
  \[
  \begin{prooftree} 
    \hypo{\text{S persistent}}
    \hypo{S \vdash P \Rightarrow P'}
    \hypo{S \vdash \{P'\} e \{v.Q'\}}
    \hypo{S \vdash \forall u. Q'[u/v] \Rightarrow Q[u/v]} 
    \infer4{S \vdash \{P\} e \{v.Q\}}
  \end{prooftree}
  \]  
  \begin{itemize}
    \item It states that we can strengthen the precondition and weaken the postcondition
  	\item The context must be persistent
  	\item This rules is often used and it is often used implicitly
  \end{itemize}
\end{itemize}
%%% Local Variables:
%%% mode: latex
%%% TeX-master: "pav-noter"
%%% End:

\subsection{Counter module}
\subsubsection{Spec version 1}
\begin{itemize}
	\item The simplest way of writing a counter module is to write the following three functions
  \begin{align*}
    \text{mk\_counter} &= \lambda \_.\text{ref}(0) \\
    \text{inc\_counter} &= \lambda x.x \leftarrow !x+1 \\
    \text{read\_counter} &= !x
  \end{align*}
  and prove the three following specifications
  \begin{align*}
    &\{\text{True}\} \; \text{mk\_counter}() \; \{v.v \hookrightarrow n+1\} \\
    &\{\ell \hookrightarrow n\} \; \text{inc\_counter} \; \ell \; \{v.v = () \land \ell \hookrightarrow n+1\} \\
    &\{\ell \hookrightarrow n\} \; \text{read\_counter} \; \ell \; \{v.v = n \land \ell \hookrightarrow n\}
  \end{align*}
  \item Problems
  \begin{itemize}
    \item Exposes the internals of the counter i.e. not so modular
    \item If a client is verified relative to this spec and the data representation has been changed then the client should be re-verified
  \end{itemize}
\end{itemize}

\subsubsection{Spec version 2}
\begin{itemize}
	\item A more abstract and modular specification of the counter module is the following which existentially quatifies over the ``counter representation predicate'' $C$ and therefore hides that the return value is a location
  \begin{align*}
    \text{mk\_counter} &= \lambda \_.\text{ref}(0) \\
    \text{inc\_counter} &= \lambda x.x \leftarrow !x+1 \\
    \text{read\_counter} &= !x
  \end{align*}
  and the following specifications
  \begin{align*}
    &\exists C: \text{Val} \rightarrow \mathbb N \rightarrow \text{Prop} \\
    &\{\text{True}\} \; \text{mk\_counter}() \; \{c.C(c,0)\} \\
    &\forall c. \{C(c,n)\} \; \text{inc\_counter} \; \ell \; \{v.v = () \land C(c,n+1)\} \\
    &\forall c. \{C(c,n)\} \; \text{read\_counter} \; \ell \; \{v.v = n \land C(c,n)\}
  \end{align*}
  \item Problems
  \begin{itemize}
  	\item Implementation code does not provide any abstraction
    \item Client of the code may directly modify the reference cell representing the counter
  \end{itemize}
\end{itemize}

\subsubsection{Spec version 3}
\begin{itemize}
	\item To encapsulate the code in a typed language one would use some form of module types / existential types
  \item In our untype language local state encapsulation is used by returning a pair of methods which have a hidden internal state variable $x$ as follows
  \begin{equation*}
    \text{counter} = \lambda\_.\text{let} \; x = \text{ref}(0) \text{ in }(\lambda\_.x \leftarrow !x + 1, \lambda \_. !x)
  \end{equation*}
  the specification of this needs to be nested hoare triples as follows 
  \begin{equation*} 
    \{\text{True} \} \text{counter}(f) \left\{
      \begin{split} 
                                                & \ell \hookrightarrow 0 * \\
        v.\exists \ell: \mathbb N \to \text{Prop}. & \forall n. \{\ell \hookrightarrow n\} v.\text{inc}() \{u.u=() \land \ell \hookrightarrow (n+1)\} * \\
                                                & \forall n. \{\ell \hookrightarrow n\} v.\text{read}() \{u.u=n \land \ell \hookrightarrow n\}  \\
      \end{split}
      \right\}
  \end{equation*}
  where $v.\text{inc}$ is used in place of $\pi_1 v$ and $v.\text{read}$ is used in place of $\pi_2 v$ 
  \item Problems
  \begin{itemize}
  	\item It exposes that the internal state is a single location
  \end{itemize}
\end{itemize}
\subsubsection{Spec version 4}
\begin{itemize}
  \item To fix the problem from version 3 existential quantification is used of the representation predicate
  \begin{equation*}
    \text{counter} = \lambda\_.\text{let} \; x = \text{ref}(0) \text{ in }(\lambda\_.x \leftarrow !x + 1, \lambda \_. !x)
  \end{equation*}
  the specification of this needs to be nested hoare triples as follows 
  \begin{equation*} 
    \{\text{True} \} \text{counter}(f) \left\{
      \begin{split} 
                                                & C(0) * \\
        v.\exists C: \mathbb N \to \text{Prop}. & \forall n. \{C(n)\} v.\text{inc}() \{u.u=() \land C(n+1)\} * \\
                                                & \forall n. \{C(n)\} v.\text{read}() \{u.u=n \land C(n)\}  \\
      \end{split}
      \right\}
  \end{equation*}
  where $v.\text{inc}$ is used in place of $\pi_1 v$ and $v.\text{read}$ is used in place of $\pi_2 v$ 
  \item Pros
  \begin{itemize}
  	\item It is modular 
  \end{itemize}
  \item Cons
  \begin{itemize}
  	\item Code and Spec are a bit more convoluted
  \end{itemize}
\end{itemize}

\subsubsection{Proof of version 4}
\begin{itemize}
	\item The syntactic sugar for the methods is removed and the predicate $C(n) = \exists \ell . \ell \hookrightarrow n$ is used as a witness
  \[
  \{\text{True} \} \text{counter}(f) \left\{
    \begin{split} 
         & \exists \ell. \ell \hookrightarrow 0  * \\
      v. & \forall n. \{\exists \ell. \ell \hookrightarrow n\} (\pi_1 v)() \{u.u=() \land \exists \ell. \ell \hookrightarrow n+1\} * \\
         & \forall n. \{\exists \ell. \ell \hookleftarrow n\} (\pi_2 v) () \{u.u=n \land \exists \ell. \ell \hookrightarrow n\}  \\
    \end{split}
    \right\}
  \]
  \item Since the program starts with a let expression the \texttt{HT-LET} rule is used. Which is done by first proving that 
  \begin{equation*}
    \{\text{True} \} \text{ref}(0) \{v. \exists \ell. \ell \hookrightarrow 0\}
  \end{equation*}
  which is done by using $\forall$I, $\exists$I and $\Rightarrow$I together with \texttt{HT-CSQ} to get
  \begin{equation*}
    \{\text{True} \} \text{ref}(0) \{v. v \hookrightarrow 0\}
  \end{equation*}
  using the \texttt{HT-RET} rule. Then we are left with proving
  \begin{equation*}
    \{\exists \ell . \ell \hookrightarrow 0 \}\left(\lambda_{-}. \ell \leftarrow !\ell+1, \lambda_{-}. ! \ell\right) \left\{
    \begin{split} 
         & \exists \ell. \ell \hookrightarrow 0  * \\
      v. & \forall n. \{\exists \ell. \ell \hookrightarrow n\} (\pi_1 v)() \{u.u=() \land \exists \ell. \ell \hookrightarrow n+1\} * \\
         & \forall n. \{\exists \ell. \ell \hookleftarrow n\} (\pi_2 v) () \{u.u=n \land \exists \ell \hookrightarrow n\}  \\
    \end{split}
    \right\}
  \end{equation*}
  \item To prove that this holds the rule \texttt{HT-FRAME} is used and thus we get
  \begin{equation*}
    \{\exists \ell . \ell \hookrightarrow 0 \}\left(\lambda_{-}. \ell \leftarrow !\ell+1, \lambda_{-}. ! \ell\right) \left\{
    \begin{split} 
         & \exists \ell. \ell \hookrightarrow 0  * \\
      v. & \forall n. \{\exists \ell. \ell \hookrightarrow n\} (\pi_1 v)() \{u.u=() \land \exists \ell. \ell \hookrightarrow n+1\} * \\
         & \forall n. \{\exists \ell. \ell \hookleftarrow n\} (\pi_2 v) () \{u.u=n \land \exists \ell \hookrightarrow n\}  \\
    \end{split}
    \right\}
  \end{equation*}
  \item The only thing left is to prove that the two returned functions abides the specification. Thus it must proven that
  \begin{equation*}
    \forall n. \{\exists \ell. \ell \hookrightarrow n\} (\pi_1 v)() \{u.u=() \land \exists \ell. \ell \hookrightarrow n+1\}
  \end{equation*} 
  holds and
  \begin{equation*}
    \forall n. \{\exists \ell. \ell \hookrightarrow n\} (\pi_2 v) () \{u.u=n \land \exists \ell . \ell \hookrightarrow n\}
  \end{equation*}
  holds.
  \item To prove the first case \texttt{HT-PROJ} is used together with \texttt{HT-BETA} and thus we are left with proving 
  \begin{equation*}
    \forall n. \{\exists \ell. \ell \hookrightarrow n\} \ell \leftarrow !\ell +1 \{u.u=() \land \exists \ell. \ell \hookrightarrow n+1\}
  \end{equation*} 
  which can be proven using \texttt{HT-BIND-DET} multiple times together with \texttt{HT-RET} to compute $!x+1$ and then \texttt{HT-STORE} rule together with \texttt{HT-CSQ}. \smallskip \\
  \item The second case \texttt{HT-PROJ} is used together with \texttt{HT-BETA} and thus we are left with proving 
  \begin{equation*}
    \forall n. \{\exists \ell. \ell \hookrightarrow n\} ! x \{u.u=n \land \exists \ell. \ell \hookrightarrow n\}
  \end{equation*} 
  which can be proven using the \texttt{HT-LOAD} rule together with \texttt{HT-CSQ} used with $\forall$I $\exists$I, $\Rightarrow$I and \texttt{Asm} on the post condition.
  \item Since the returned function obeys the required specification, the specification of the \texttt{counter} module must be correct. 
\end{itemize}

\newpage
%%% Local Variables:
%%% mode: latex
%%% TeX-master: "pav-noter"
%%% End:
