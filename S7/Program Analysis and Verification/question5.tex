\section{Pointer Analysis}
\subsection{Allocation-Site Abstraction}
\begin{itemize}
  \item \textbf{Allocation-site abstraction} is a common choice for representing the set of possible memory cells that the pointers may point to, which is to introduce
  \begin{itemize}
  	\item An abstraction cell $X$ for program variable $X$
  	\item An abstraction cell \texttt{alloc-i} for where $i$ is a unique index for each occurrence of an $\mathtt{alloc}$ operation
  	\item Each abstract cell represents the set of cells at runtime that are allocated to the same source location
  	\item Cells is used to denote the set of abstract cells for the given program
  	\item Locs is used to denote the set of abstract location of the cells which is written as $\& c \in \text{Locs}$ for every $c \in \text{Cells}$ 
  \end{itemize}
	
\end{itemize}

\subsection{Andersen's Algorithm}
\begin{itemize}
  \item It is an approach to points-to analysis
  \item For each cell $c$ constraint variable $\co{c}$ is introduced ranging over sets of locations
  \item Uses the cubic algorithm to compute a solution
  \item The analysis assumes that the program has been normalized so that every pointer operation is one of these six kinds
  \begin{itemize}
  	\item $X = \mathtt{alloc}~P$ where $P$ is $\mathtt{null}$ or an integer constant
  	\item $X_1 = \& X_2$
  	\item $X_1 = X_2$
  	\item $X_1 = * X_2$
  	\item $*X_1 = X_2$
  	\item $X = \mathtt{null}$
  \end{itemize}
  \item The following constraints are generated for each of these pointer operations
  \begin{align*} 
    X = \mathtt{alloc} \ P &: \quad  \& \mathtt{alloc-i} \in \co{X} \\
    X_1 = \&X_2 &: \quad  \& X_2 \in \co{X_1} \\
    X_1 = X_2 &: \quad \co{X_2} \subseteq \co{X_1} \\
    X_1 = *X_2 &: \quad \&\alpha \in \co{X_2} \Rightarrow \co{\alpha} \subseteq \co{X_1} \\
    *X_1 = X_2 &: \quad \& \alpha \in \co{X_1} \Rightarrow \co{X_2} \subseteq \co{\alpha}
  \end{align*}
  \item The resulting points to function is defined as:
  \begin{equation*}
    \mathtt{pt}(p) = \co{p}
  \end{equation*}	
\end{itemize}

\subsection{Steensgaard's Algorithm}
\begin{itemize}
  \item It is an approach to points-to analysis
  \item It is a coarsers analysis which can be expressed using term unification
  \item A term variable $\co{c}$ is used for every cell $c$
  \item A term constructor $\& t$ is used to represent a pointer to $t$
  \item Each $\alpha$ denotes a fresh term variable
  \item Term constructors satisfy the general term equality axiom:
  \begin{equation*}
    \& \alpha_1 = \& \alpha_2 \Rightarrow \alpha_1 = \alpha_2
  \end{equation*}
  \item The following constraints are generated for each of these pointer operations
  \begin{align*} 
    X = \mathtt{alloc} \ P &: \quad  \co{X} = \& \mathtt{alloc-i}\\
    X_1 = \&X_2 &: \quad \co{X_1} = \& \co{\& X_2} \\
    X_1 = X_2 &: \quad \co{X_2} = \co{X_1} \\
    X_1 = *X_2 &: \quad \co{X_2} = \&\alpha \land \subseteq \co{X_1} = \alpha \\
    *X_1 = X_2 &: \quad \co{X_1} = \&\alpha \land \subseteq \co{X_2} = \alpha
  \end{align*}
  \item The resulting points to function is defined as:
  \begin{equation*}
    \mathtt{pt}(p) = \left\{t \in Cells \mid \co{p} = \co{t}\right\}
  \end{equation*}	
\end{itemize}

\subsection{Interprocedural Points-To Analysis}
\begin{itemize}
  \item CFG and points-to analysis may depend on each other and therefore one should perform the two simultaneously
  \item All function calls should be normalized to the form
  \begin{equation*}
    X = X'(X_1, \dots, X_n)
  \end{equation*} 
  i.e. all involved expressions are variables and all return expression are assumed to be just variables
  \item Andersen's algorithm are very similar to cfg analysis and therefore it can be extended with the appropriate constraints
  \begin{itemize}
    \item A reference to a constant function $f$ generates the constraint
    \begin{equation*}
     f \in \co{f} 
    \end{equation*} 
    \item The computed function call generates the constraint
    \begin{equation*}
      f \in \co{X'} \Rightarrow (\co{X_1} \subseteq \co{X_1'} \land \cdots \land \co{X_n} \subseteq \co{X_n'} \land \co{X''} \subseteq \co{X})
    \end{equation*}
    for every occurrence of a function with $n$ parameters
    \begin{equation*}
      f(X_1', \dots, X_n) \{\dots \mathtt{return} \; X''; \}
    \end{equation*}
  \end{itemize}
\end{itemize}

\subsection{Flow-Sensitive Points-to Analysis}
\begin{itemize}
  \item It can e.g. be used to discover interesting heap structures
  \item A lattice of \textbf{point-to graphs} is used
  \begin{itemize}
  	\item It is directed graphs in which the nodes are the abstract cells of for the given graph and the edges correspond to possible pointers
  	\item They are ordered by inclusion of their sets to edges
  	\item $\bot$ is the graph without edges and $\top$ is the fully connected graph
  	\item i.e. the lattice for abstract states is 
  \end{itemize}
  \begin{equation*}
    States = 2^{Cells \times Cells}
  \end{equation*}
  \item For every CFG node $v$ a constraint variable $\co{v}$ is introduced denoting a point-to graph that describes all possible stores at that program point
  \item Constraints:
  \begin{align*}
    X = \mathtt{alloc} \; P &: \quad \co{v} = \text{JOIN}(v) \downarrow X \cup \{(X, \text{\texttt{alloc-i}})\} \\
    X_1 = \& X_2 &: \quad \co{v} = \text{JOIN}(v) \downarrow X_1 \cup \{(X_1, X_2)\} \\
    X_1 = X_2 &: \quad \co{v} = \text{assign}(\text{JOIN}(v), X_1, X_2)\\
    X_1 = *X_2 &: \quad \co{v} = \text{load}(\text{JOIN}(v), X_1, X_2)\\
    *X_1 = X_2 &: \quad \co{v} = \text{store}(\text{JOIN}(v), X_1, X_2)\\
    X = \text{null} &: \quad \co{v} = \text{JOIN}(v)\downarrow X
  \end{align*}
  and for all other nodes:
  \begin{align*}
    \co{v} = \text{JOIN}(V)
  \end{align*}
  where
  \begin{align*}
                 \text{JOIN}(v) &= \cup_{w \in \text{pred}(v)} \co{w} \\
            \sigma \downarrow x &= \{(s,t) \in \sigma \mid s \neq x\} \\
    \text{assign}(\sigma, x, y) &= \sigma \downarrow x \cup \{(x,t) \mid (y,t) \in \sigma \}\\
      \text{load}(\sigma, x, y) &= \sigma \downarrow x \cup \{(x,t) \mid (y,s) \in \sigma, (s,t) \in \sigma \}\\
      \text{store}(\sigma, x, y) &= \sigma \cup \{(s,t) \mid (x,s) \in \sigma, (y,t) \in \sigma \}
  \end{align*}
  \item The analysis also computes a flow sensitive points-to map that for each program point $v$ is defined by
  \begin{equation*}
    pt(p) = \{t \mid (p,t) \in \co{v}\}
  \end{equation*}
\end{itemize}
  
\newpage
%%% Local Variables:
%%% mode: latex
%%% TeX-master: "pav-noter"
%%% End:
