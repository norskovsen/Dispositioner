\section{Public-key Crypto based on Discrete Log and LWE}
\subsection{Public-key Cryptosystem Definition}
\begin{itemize}
  \item A trapdoor oneway function is an injective $f: X \to Y$ where are $X,Y$ are finite sets is a function such that
  \begin{itemize}
    \item Computing $f(x)$ from input $x$ is easy
    \item Computing $x$ from $f(x)$ for a random $x \in X$ is infeasible
    \item $f$ might come with a trapdoor $t_f$ such that given this extra information it would be easy to compute $x$ from $f(x)$
  \end{itemize}
  \item Pairs $(f,t_f)$ of function and trapdoor could be a public key encryption system
  \begin{itemize}
    \item $f$ could serve as the public key
    \item Anyone could encrypt $x \in X$ to ciphertext $y = f(x)$
    \item Only the party who knows $t_f$ is able to recover $x$ from $y$
  \end{itemize}

  \item \textbf{Definition 6.1} A public-key cryptosystem consists of 3 algorithms $(G,E,D)$, satisfying the following:
  \begin{itemize}
  	\item $G$ algorithm for generating keys:
    \begin{itemize}
  		\item It is probabilistic
  		\item Takes security parameter $k$ as input
  		\item Always outputs a pair of keys $(pk,sk)$ (the public and secret key)
  		\item It is assumed that the public key contains a description of
      \begin{itemize}
  			\item $\mathcal P$, the set of plaintexts
  			\item $\mathcal C$, the set of ciphertexts
      \end{itemize}
  		\item $\mathcal P$ and $\mathcal C$ does not have to be the same for every key
    \end{itemize}
  	\item $E$ an algorithm for encryption
    \begin{itemize}
  		\item It takes as input $pk$ and $x \in \mathcal P$
  		\item It produces as output $E_{pk}(x) \in \mathcal C$
  		\item It may be probabilistic i.e. the ciphertext will have a probability distribution that is determined from $x$ and $K$
    \end{itemize}
  	\item $D$ algorithm for decryption
    \begin{itemize}
  		\item It takes as input $sk, y \in \mathcal C$ and produces as output $D_{sk}(y) \in \mathcal P$
  		\item It is allowed to be probabilistic but is in most cases deterministic
    \end{itemize}
  	\item It is required for any $x \in \mathcal P$ that $x = D_{sk}(E_{pk}(x))$
  \end{itemize}
\end{itemize}
%%% Local Variables:
%%% mode: latex
%%% TeX-master: "crypto-noter"
%%% End:


\subsection{Computational problems}
\begin{itemize}
  \item \textbf{The discrete log (DL) problem} Given a group $G$, generator $\alpha$ and $\beta \in G$ find an integer such that $\alpha^a = \beta$
	\item It is in many groups hard e.g. $\mathbb Z_p^*$
  \item \textbf{The Diffie-Hellman (DH) problem} Given a group $G$, generator $\alpha$, and $\alpha^a$, $\alpha^b$ where $a,b$ are randomly and independently chosen from $Z_t$ compute $\alpha^{ab}$
  \item \textbf{Lemma 8.1} The DH problem is no harder than the DL problem

  \item \textbf{The Decisional Diffie-Hellman (DDH) problem} Given a group $G$, generator $\alpha$ and $\alpha^a, \alpha^b, \alpha^c$ where $a,b$ are randomly and independently chosen from $Z_t$ guess whether which of the following cases you are in
  \begin{itemize}
  	\item $c$ is chosen as $c=ab$
  	\item $c$ is chosen uniformly at random from $Z_t$
  \end{itemize}
  \item \textbf{Lemma 8.2} The DDH problem is no harder than the DH problem
  \item A \textbf{group generator} /GGen/ is a efficient probabilistic algorithm which takes as input an integer $k$ and outputs a group $G$ and an element $\alpha \in G$ that generates $G$
  \begin{itemize}
  	\item $k$ controls the size of the group generated
  \end{itemize}
  \item \textbf{Definition 8.3} Consider the following experiment with an algorithm $A$:
  \begin{itemize}
  	\item Run \textit{GGen} on input $k$ to get $G$ and $\alpha$
  	\item Choose $a$ at random in $Z_t$ and give $A$ input $G, \alpha, \alpha^a$
  	\item The DL problem is said to be hard (with respect to \textit{GGen}) if for any polynomial time in $k$ algorithm $A$, the probability that $A$ outputs $a$ is negligible in $k$
  \end{itemize}
  \item \textbf{Definition 8.4} Consider the following experiment with an algorithm $A$:
  \begin{itemize}
  	\item Run \textit{GGen} on input $k$ to get $G$ and $\alpha$
  	\item Choose $a,b$ at random in $Z_t$ and give $A$ input specification of $G, \alpha, \alpha^a, \alpha^b$
  	\item The DH problem is said to be hard (with respect to \textit{GGen}) if for any polynomial time in $k$ algorithm $A$, the probability that $A$ outputs $\alpha^{ab}$ is negligible in $k$
  \end{itemize}
  \item \textbf{Definition 8.5} Consider the following experiment with an algorithm $A$:
  \begin{itemize}
  	\item Run \textit{GGen} on input $k$ to get $G$ and $\alpha$
  	\item Choose $a,b$ at random in $Z_t$
  	\item One can be in one of the following two cases:
    \begin{enumerate}[label=\alph*)]
  		\item In the "real" case, set $c=ab$
  		\item In the "ideal" case choose $c$ at random from $Z_t$
    \end{enumerate}
  	\item Give $A$ input specification of $G,\alpha, \alpha^a, \alpha^b, \alpha^c$.
  	\item $A$ outputs one bit i.e. a guess at whether one in the real or the ideal case
  	\item Let $p_{A,0}(k)$ be the probability that $A$ outputs $1$ in the real case
  	\item Let $p_{A,1}(k)$ be the probability that $A$ outputs $1$ in the real case
  	\item The advantage of $A$ is defined to be 
    \begin{equation*}
      Adv_A(k) = |p_{A,0}(k) - p_{A,1}(k)|  
    \end{equation*}
  	\item The DH problem is said to be hard (with respect to \textit{GGen}) if for any polynomial time in $k$ algorithm $A$, $Adv_A(k)$ is negligible in $k$
  \end{itemize}
\end{itemize}

\subsection{The El Gamal Cryptosystem}
\begin{itemize}
  \item Given to parties $A$ and $B$ that want to exchange a secret given an already agreed (in public) group $G$ and a generator $\alpha$:
  \begin{enumerate}
  	\item $A$ chooses $s_A$ at random in $Z_t$ and $B$ chooses $s_B$ at random in $Z_t$
  	\item $A$ sends $y_A = \alpha^{s_A}$ to $B$ and $B$ sends $y_B = \alpha^{S_B}$ to $A$
  	\item $A$ computes $y_B^{s_A}$ and $B$ computes $y_A^{S_B}$
  \end{enumerate}

  \item \textbf{El Gamal cryptosystem (general version)}
  \begin{itemize}
  	\item \textbf{Key generation} On input security parameter $k$
    \begin{itemize}
  		\item Run \textit{GGen} on input $k$ to obtain specification of a group $G$ and generator $\alpha$
  		\item Choose $a$ at random from $Z_t$
  		\item The public key is the specification of $G$ and $\beta = \alpha^a$
  		\item The secret key is $a$
  		\item The plaintext space is $G$ and the ciphertext space is $G \times G$
    \end{itemize}
  	\item \textbf{Encryption} To encrypt $m \in G$ choose $r$ at random in $Z_t$, the ciphertext is $(\alpha^r, \beta^rm)$
  	\item \textbf{Decryption} To decrypt ciphertext $(c,d)$ compute $c^{-a}d$
  \end{itemize}

  \item \textbf{Lemma 8.6} The problem of decrypting an El Gamal ciphertext (without the secret key) is equivalent to solving the DH problem

  \item \textbf{Theorem 8.7} If the DDH problem is hard (w.r.t \textit{GGen)}, then the El Gamal cryptosystem is CPA \smallskip \\
  \textbf{Proof}
  \begin{itemize}
   \item It is assumed for contradiction, that there exists some adversary \texttt{Adv} that plays the CPA security game and has advantage at least $\epsilon$ for the El Gamal cryptosystem
   \item Using the adversary one can create an algorithm $B$ can distinguish between ``c is random'' and ``c=ab'' with advantage at least $\epsilon$.
   \item Let $G, \alpha, \alpha^a, \alpha^b, \alpha^c$ be the input to the algorithm $B$.
   \item Then set the public key to be $\beta = \alpha^a$ and give this to the adversary together with the specification of the group $G$
   \item When the adversary \texttt{Adv} ask for the encryption of a message $m \in G$ send the ciphertext $(\alpha ^b, \alpha^c m)$. 
   \begin{itemize}
     \item If $c=ab$ this would be a correct ciphertext where $r=b$ since
     \begin{equation*}
       (\alpha ^b, \alpha^c m) = (\alpha^b, \alpha^{ab}m) = (\alpha^b, \beta^b m) = E_{\beta} (m)
     \end{equation*} 
     \item If $c$ was random it was the ciphertext of $r' = \alpha^{c+d-ab}$ for $m = \alpha^{d}$ since 
     \begin{equation*}
       (\alpha ^b, \alpha^c m) = (\alpha^b, \alpha^{c}\alpha^d) = (\alpha^b, \alpha^{c+d}) = (\alpha^b, \alpha^{ab}\alpha^{c+d-ab}) = (\alpha^b, \beta^br') = E_{\beta}(r')
     \end{equation*}
     Since $c$ is random $r'$ is random since it can be determined from $c$ and the correct answer for the adversary \texttt{Adv} is ideal.
   \end{itemize}
   \item When the Adversary \texttt{Adv} outputs ideal output ``c is random'' and if it outputs real output ``c=ab''
   \item This achieves an advantage of at least $\epsilon$ and if this is not negligible, then DDH is not hard which is a contradiction
   \item Thereby the El Gamal cryptosystem must be CPA secure for some negligible $\epsilon$
  \end{itemize}
\end{itemize}

\subsection{Example groups}
\subsubsection{$\mathbb Z_p^*$}
\begin{itemize}
  \item \textbf{Lemma 8.8} $\alpha \in \mathbb Z_p^*$ is a generator if and only if $\alpha^{(p-1)/q} \neq 1$ for every prime $q$ that divides $p-1$
  \item To generate a prime $p$ where the factorisation of $p-1$ is known one can do the following
  \begin{itemize}
  	\item Choose a random prime $q$ set $p=2q + 1$ and test whether $p$ is a prime
  	\item If not choose a new $q$
  	\item When this succeeds the factor of $p-1$ is known i.e. $2$ and $q$
  	\item It is expected to generate $O(k)$ values of $q$ before $p$ happens to be prime
  	\item A prime $p$ generated in this way is sometimes called a safe prime
  \end{itemize}

  \item If $G= \mathbb Z_p^*$ in El Gamal one will never get a CPA-secure cryptosystem

  \item \textbf{Lemma 8.9} Let $\alpha$ be a generator of $\mathbb Z^*_p$. Then $(\alpha^i)^{(p-1)/2} \text{ mod } p = 1$ if and only if $i$ is even and is $-1$ otherwise

  \item \textbf{Lemma 8.10} In the group $\mathbb Z_p^*$, the DDH problem is not hard \smallskip \\
  \textbf{Proof}
  \begin{itemize}
  	\item The adversary is given $\alpha, \alpha^a, \alpha^b, \alpha^c$ 
    \item Using lemma 8.9 one can compute whether $a,b$ are even or odd and therefore the parity of $ab$
    \item He can also find the parity of $c$
    \item If $c=ab$ then the two parities always match
    \item If $c$ is randomly and independently chosen then they will match with probability $1/2$
    \item If the adversary compares the parities of $ab$ and $c$ and if they equal then he guess that he is in case $ab = c$ 
    \item Otherwise he guesses that $c$ is random
    \item This achieves advantage $1-1/2 = 1/2$ which is not negligible
  \end{itemize}
\end{itemize}

\subsubsection{Prime Order Subgroups of $\mathbb Z_p^*$}
\begin{itemize}
  \item To get CPA-security one needs to use, instead of $\mathbb Z_p^*$ a subgroup of large prime order
  \begin{itemize}
  	\item This can be done using safe primes
  \end{itemize}
  \item For every prime $q$ that divides $p-1$ for prime $p$ there is exactly one subgroup $G$ of order $q$
  \begin{itemize}
  	\item For a safe prime if $\alpha_0$ generates all of $\mathbb Z_p^*$ then $\alpha = \alpha_0^2$ generates this subgroup
  \end{itemize}

  \item Algorithm for generating $G$ and $\alpha$:
  \begin{itemize}
	  \item On input $k$ generate a $k$ bit prime $p$ where $p = 2q +1$ and generator $\alpha _0$ of $\mathbb Z _ p^*$
	  \item Let $G$ be the subgroup of order $q$, and set $\alpha = \alpha_0^2$. Output $p,q,\alpha$
  \end{itemize}

  \item To be able to encrypt and decrypt any number in $Z_q$ one does the following
  \begin{itemize}
  	\item \textbf{Encryption} On input $x \in Z_q$ set $y=x+1$ and compute $y^{(p-1)/2} \text{ mod } p$
    \begin{itemize}
  		\item If this is $1$ output $y$ else output $-y \text{ mod } p$
    \end{itemize}
  	\item \textbf{Decryption} On input $g \in G \subset \mathbb Z_p^*$ test if $g \leq q$.
    \begin{itemize}
  		\item If so set $y=g$ else set $y = -g \text{ mod } p$.
  		\item Output $y-1$
    \end{itemize}
  \end{itemize}
\end{itemize}

\subsection{Public-Key Encryption Based on LWE}
\begin{itemize}
  \item \textbf{Learning with errors (LWE) problem} is defined over a finite field $F_q$ for a prime $q$
  \begin{itemize}
  	\item All computations are done modulo $q$
  	\item A secret vector $\mathbf{s} \in F_q^n$ are involved
  	\item The problem is: Given random $\left\{\mathbf{a}_{i} \in F_{q}^{n}\right\}_{i=1}^{m}$ and $\left\{\mathbf{a}_{i} \cdot \mathbf{s}+e_{i}\right\}_{i=1}^{m}$ where the $e_i$'s are random by numerical small. The goal is to find $\mathbf s$
  	\item $e_i$'s being random and numerical small always means that they are much smaller than $q$
    \begin{itemize}
  		\item $e_i$ can be e.g. uniformly chosen in the interval $[-\sqrt{q}, \sqrt{q}]$
    \end{itemize}
  	\item It is assumed that the $e_i$'s are chosen with a distribution $D_e$ with the following property: \textit{even if one sample $m$ values distributed according to $D_e$ take numeric value ad sum them, the result will be smaller than $q/4$ with overwhelming probability}
  \end{itemize}

  \item Decision version of LWE
  \begin{itemize}
  	\item One are given random $\left\{\mathbf{a}_{i} \in F_{q}^{n}\right\}_{i=1}^{m}$ and either $\left\{\mathbf{a}_{i} \cdot \mathbf{s}+e_{i}\right\}_{i=1}^{m}$ or $\left\{u_{i}\right\}_{i=1}^{m}$ where the $u_{i}$ are uniformly random
  	\item The goal is to decide which case you are in
  	\item The decision LWE is hard if any PPT algorithm can only distinguish the two cases with negligible advantage in $n$
  \end{itemize}

  \item Secret key based on LWE
  \begin{itemize}
  	\item The secret key is $\mathbf{s}$
  	\item To encrypt a bit $w \in \{0,1\}$ one will send the ciphertext: $(\mathbf{a}, \mathbf{a} \cdot \mathbf{s}+e+\lceil q / 2\rceil w)$
  	\item To decrypt the cipher text $(\mathbf{u},v)$ the receiver can compute $v-\mathbf{u} \cdot \mathbf{s}$
    \begin{itemize}
  		\item If Clearly, if $e<<q$ then this value will be close to 0 if $w=0$ and close to $|q / 2|$ if $w=1$
    \end{itemize}
  \end{itemize}

  \item Public key based on LWE
  \begin{itemize}
  	\item \textbf{Key Generation}
    \begin{itemize}
    	\item \textbf{Secret key:} random vector $s \in F_{q}^{n}$ 
      \item \textbf{Public key:} $\left\{c_{i}\right\}_{i=1}^{m}$ where $c_{i}=\left(\mathbf{a}_{i}, \mathbf{a}_{i} \cdot \mathbf{s}+e_{i}\right)$ where $\mathbf{a}_{i}$ are uniformly random and the $e_{i}$ are chosen according to $D_{e}$
    \end{itemize}
  	\item \textbf{Encryption}
    \begin{itemize}
    	\item Message is a bit $w$
      \item Choose random bits $b_{1}, \ldots, b_{m}$ and then the ciphertext is $\sum_{i=1}^{m} b_{i} c_{i}+(0,\lceil q / 2\rceil w)$
    \end{itemize}
  	\item \textbf{Decryption} To decrypt $(\mathbf{u}, v)$ compute $v-\mathbf{s} \cdot \mathbf{u}$
    \begin{itemize}
    	\item Output 0 if this value is closer to 0 than it is to $\lceil q / 2\rceil$
      \item Output 1 otherwise.
    \end{itemize}
  \end{itemize}

  \item \textbf{Lemma 9.1} If $m \geq n + (n+1) \log_2(q)$, then the following two distributions can be distinguished with only negligible in $n$ advantage (even by an unbounded adversary)
  \begin{align*}
    \{\mathbf{a}_i, u_i\}_{i=1}^m, \sum_{i=1}^m b_i(\mathbf{a_i},u_i) \\
    \{\mathbf{a}_i, u_i\}_{i=1}^m, \mathbf{r}
  \end{align*}
  where $\mathbf{r} \in F_q^{n+1}$ is chosen uniformly and independently of anything else
  \begin{itemize}
  	\item Thereby $\sum_{i=1}^m b_i(\mathbf{a_i},u_i)$ is essentially uniformly random even if one are given $\{\mathbf{a}_i, u_i\}_{i=1}^m$ but not the $b_i$'s
  \end{itemize}
  \item This basic form of cryptosystem is quire inefficient since to get reasonable security one needs $q$ to be around 1000 and $n$ to be a few hundred
\end{itemize}

\newpage

%%% Local Variables:
%%% mode: latex
%%% TeX-master: "crypto-noter"
%%% End:
