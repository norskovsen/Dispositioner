\section{Actively secure MPC}

\subsection{$\F_{\COM}$ definition}
\begin{itemize}
  \item It is assumed that each player $\P_i$ can commit to a value $a \in \mathbb F$ while keeping the choice secret which they might choose to reveal later

  \item Commitments are modelled by assuming an ideal functionality $\FCOM$
  \begin{itemize}
  	\item To commit, one simply sends $a$ to $\FCOM$, which will keep $a$ until $\P_i$ asks to have it revealed
 		\item Formally it is assumed that $\FCOM$ is equipped with commands \texttt{commit} and \texttt{open} as described in the box
  \end{itemize}

	\item The implementation of $\FCOM$ requires that all honest players take part actively 
  \begin{itemize}
    \item It is required that all honest players in a given round send the same commands to $\FCOM$ in order for the command to be executed
  \end{itemize}

	\item $\FCOM$ will do a complete breakdown if
  \begin{itemize}
		\item It is not used as intended by the honest parties
		\item An honest party commits under the same $cid$ twice
		\item An honest party use the same $cid_3$ twice in an addition or a multiplication command
		\item The honest parties use some $cid_1$ or $cid_2$ that has not been defined yet
  \end{itemize}

	\item The following convention is used for specifying $\FCOM$
  \begin{itemize}
		\item When a command gives a particular output, then this output is not delivered immediately.
    \begin{itemize}
			\item It will be specified via $\mathtt{COM}.\mathtt{infl}$ in which round to deliver, and in that round, the output to all parties will be given
    \end{itemize}
		\item If a party $\P_i$ is corrupted in between a command given to $\FCOM$ and the output delivery round, then it is allowed that the input of $\P_i$ to the command is changed via $\COM.\infl$ and the outputs will be recomputed accordingly
		\item When a command leaks a value, then the value is output on $\COM.\leak$ in the round where inputs to the commands are given, not in the delivery round
  \end{itemize}

	\item The final part of the description of $\FCOM$ includes to \textbf{advanced manipulation} commands
  \begin{itemize}
		\item The \texttt{transfer} command transfer a committed value $a$ from a party $\P_i$ to another $\P_j$
    \begin{itemize}
			 \item Equivalent to the $P_j$ having committed to $a$ where $\P_i$ knows the value of $a$
    \end{itemize}
		\item The \texttt{mult} command is used by a player to create, given the commitments to $a_1$ and $a_2$ a new commitment that is guaranteed to contain $a_1a_2$
  \end{itemize}

	\item The symbol $\commit{\cdot}{i}$ denoted a variable in which $\FCOM$ keeps a committed value received from player $\P_i$
  \begin{itemize}
		\item $\commit{a}{i}$ means that player $\P_i$ have committed to $a$
  \end{itemize}

	\item The following notation is used as a shorthand for commands
  \begin{itemize}
		\item $\commit{a}{i} \leftarrow a$: the \texttt{commit} command is executed to let $\P_i$ commit to $a$
		\item $\commit{a}{i} \Leftarrow a$: the \texttt{pubcommit} command is used to force $\P_i$ commit to $a$
		\item $a \leftarrow \commit{a}{i}$: the \texttt{open} command is executed to let all parties learn $a$
		\item $(\P_j)a \leftarrow \commit{a}{i}$: the designated \texttt{open} command is executed to let $\P_j$ learn $a$
		\item $\commit{a}{j} \leftarrow \commit{a}{i}$: the \texttt{transfer} command from $\P_i$ to $\P_j$
		\item $\commit{a_3}{i} \leftarrow \commit{a_1}{i}\commit{a_2}{i}$: the \texttt{mult} command
		\item $\commit{a_3}{i} \leftarrow \commit{a_1}{i} + \commit{a_2}{i}$: the \texttt{add} command
		\item $\commit{a_3}{i} \leftarrow \alpha\commit{a_2}{i}$: the \texttt{mult} command
  \end{itemize}

	\item One can ask $\FCOM$ to manipulate committed values as long as the involved variables belong to the same party

  \item The advanced manipulation commands can be implemented from the basic set of operations i.e. given a functionality $\F_{\COM-\mathtt{SIMPLE}}$ that has all the commands of $\FCOM$ except the advanced manipulation commands one can build protocols that will implement $\FCOM$ when given access to $\F_{\COM-\mathtt{SIMPLE}}$
\end{itemize}

\subsection{Implementations of the transfer command}
\begin{protocol}{Protocol Transfer}{transfer}
  If any command fails in steps 1-3, then either $P_i$ or $P_j$ is corrupt and in this case go to step 4
  \begin{enumerate}
  	\item $(P_j)a \leftarrow \commit{a}{i}$
    \item $\commit{a}{j} \leftarrow a$
    \item For $k=1, \dots, n$ do
    \begin{enumerate}
    	\item $P_i$ picks a uniformly random $r \in_R \F$ and sends it privately to $P_j$
      \item Execute $\commit{r}{i} \leftarrow r$ and $\commit{r}{j} \leftarrow r$
      \item $P_k$ broadcasts a random challenge $e \in \F$
      \item The parties execute $\commit{s}{i} \leftarrow e \commit{a}{i} + \commit{r}{i}$ and $\commit{s}{j}  \leftarrow e \commit{a}{j} + \commit{r}{j}$
      \item $s \leftarrow \commit{s}{i}$ and $s' \leftarrow \commit{s}{j}$
      \item If $s \neq s'$, all players goto step 4
    \end{enumerate}
    If all $n$ iterations of the loop were successful, parties output \texttt{success}, except $\P_j$, who outputs $a$
    \item The ``exception handler'' only executed if either $\P_j$ or $\P_i$ is corrupt:
    \begin{enumerate}
      \item Execute $a \leftarrow \commit{a}{i}$. If this fails, all parties output \texttt{fail} otherwise, do $\commit{a}{j} \Leftarrow a$.
      \item Parties output \texttt{success}, except $\P_j$, who outputs $a$
    \end{enumerate}
  \end{enumerate}
\end{protocol}

\begin{protocol}{Protocol Perfect Transfer}{perfect-transfer}
  If any command fails in steps 1-4, then either $\P_i$ or $\P_j$ is corrupt. In this case, go to step 5
  \begin{enumerate}
    \item $(\P_j)a \leftarrow \commit{a}{i}$
    \item $\commit{a}{j} \leftarrow a$
    \item To check that $\P_i$ and $\P_j$ are committed to the same value the following is done:
    \begin{enumerate}
    	\item $\P_i$ selects a polynomial $f(\X) = a + \alpha_1 \X + \cdots + \alpha_t \X^t$ for uniformly random $\alpha_1, \dots, \alpha_t \in_R \mathbb F$ 
      \item All players compute for $k=1, \dots n$,
      \[
        \commit{f(k)}{i} = \commit{a}{i} + k \commit{\alpha_1}{i} + \cdots + k^t \commit{\alpha_t}{i}
      \]
      using the \texttt{add} and \texttt{mult} commands
      \item  $(P_k)f(k) \leftarrow \commit{f(k)}{i}$
      \item $\P_i$ sends $\alpha_1, \dots, \alpha_t$ privately to $P$ 
      \item Execute $\commit{\alpha_l}{j} \leftarrow \alpha_l$ for $l=1,\dots,t$ the same way as earlier
      \item Compute $\commit{f(k)}{j}$ and execute $(\P_k)f(k) \leftarrow \commit{f(k)}{j}$
    \end{enumerate}
    \item For $k=1, \dots,n$, $\P_k$ compares the two values that were opened toward him and if they agree, he broadcasts \texttt{accepts}; otherwise, he broadcasts \texttt{reject}
    \item This is the ``exception handler'' this is executed if it is known that $\P_j$ or $\P_i$ is corrupt:
    \begin{enumerate}
    	\item $a \leftarrow \commit{a}{i}$ if this fails all parties output \texttt{fail} otherwise $\commit{a}{j} \Leftarrow a$ 
      \item Parties output \texttt{success} except $\P_j$ who outputs $a$
    \end{enumerate}
  \end{enumerate}
\end{protocol}

\begin{itemize}
  \item \textbf{Protocol Transfer} (\autoref{prot:transfer}) is only statistically secure for any $t < n$
  \begin{itemize}
  	\item It assumes that $1/|\mathbb{F}|$ is negligible in the security parameter and if not the step 2 is repeated $u$ times such that $(1/|\mathbb{F})^u$ is negligible.
  \end{itemize}
  \item \textbf{Protocol Perfect Transfer} (\autoref{prot:perfect-transfer}) is perfectly secure but requires $t < n/2$
  \item The idea behind the protocols is that party $\P_i$ reveals to party $\P_j$ and $P_j$ commits to this. Then they prove to the other parties that what the have committed to are then same

\end{itemize}

\subsection{Implementations of the mult command}
\begin{protocol}{Protocol Commitment Multiplication}{commit-mult}
  If any command used udring this protocol fails, all players output \texttt{fail}
  \begin{enumerate}
  	\item $\P_i$: $\commit{c}{i} \leftarrow ab$.
    \item For $k=1,\dots,n$ do
    \begin{enumerate}
    	\item $\P_i$ choose a uniformly random $\alpha \in_R \mathbb F$
      \item $\commit{\alpha}{i}$; $\commit{\gamma}{i} \leftarrow \alpha b$.
      \item $\P_k$ broadcasts a uniformly random challenge $e \in_R \mathbb F$
      \item $\commit{A}{i} \leftarrow e \commit{a}{i} + \commit{\alpha}{i}$; $A \leftarrow \commit{A}{i}$
      \item $\commit{D}{i} \leftarrow A \commit{b}{i} - e \commit{c}{i} - \commit{\gamma}{i}$; $D \leftarrow \commit{D}{i}$.
      \item If $D \neq 0$, all players output \texttt{fail}
    \end{enumerate}
  \item If this step is used, no command failed during the protocol i.e. all the $D$ values were $0$ and all players output \texttt{success}.
  \end{enumerate}
\end{protocol}

\begin{protocol}{Protocol Perfect Commitment Multiplication}{commit-mult-perf}
  If any command used during this protocol fails, all players output \texttt{fail}
  \begin{enumerate}
  	\item $\commit{c}{i} \leftarrow ab$
    \item $\P_i$ chooses polynomials $f_a(\X) = a + \alpha_1 \X + \cdots + \alpha_tX^t$ and $f_b(\X) = b + \beta_1 \X + \cdots + \beta_tX^t$ for random $\alpha_j, \beta_j \in_R \mathbb F$. He or she computes $h(\X)=f_a(\X)f_b(\X)$ and writes $h(\X)$ as $h(\X) = c + \gamma_1 \X + \cdots + \gamma_{2t} \X^{2t}$. Then establish commitments as follows
    \begin{align*}
      \commit{\alpha_j}i &\leftarrow \alpha_j \quad \text{for } j=1, \dots, t \\
      \commit{\beta_j}i &\leftarrow \beta_j \quad \text{for } j=1, \dots, t \\
      \commit{\gamma_j}i &\leftarrow \gamma_j \quad \text{for } j=1, \dots, 2t
    \end{align*}
    \item The \texttt{add} and \texttt{mult} commands are invoked to compute, for $k=1 ,\dots, n$,
    \begin{align*} 
      \commit{f_a(k)}i &= \commit{a}i + k \commit{\alpha_1}i + \cdots + k^t \commit{\alpha_t}i \\
      \commit{f_b(k)}i &= \commit{b}i + k \commit{\beta_1}i + \cdots + k^t \commit{\beta_t}i \\
      \commit{h(k)}i &= \commit{c}i + k \commit{\gamma_1}i + \cdots + k^{2t} \commit{\gamma_{2t}}i
    \end{align*} 
    Then $(P_k)a_k \leftarrow \commit{f_a(k)}i$, $(\P_k)b_k \leftarrow \commit{f_b(k)}i$ and $(\P_k)c_k \leftarrow \commit{h(k)}i$ are executed
    \item For $k=1, \dots,n$, do: $\P_k$ checks that $a_kb_k = c_k$ and broadcasts \texttt{accept} if this is the case; otherwise he broadcast \texttt{reject}
    \item For each $\P_k$ who said \texttt{reject}, do $a_k \leftarrow \commit{f_a(k)}i$, $b_k \leftarrow \commit{f_b(k)}i$, and $c_k \leftarrow \commit{h(k)}i$; all players check whether $a_kb_k = c_k$ holds. If this is not the case, all players output \texttt{fail}.
    \item If we arrive here, no command failed during the protocol, and thus all relations $a_kb_k = c_k$ that were checked hold. All players output \texttt{success}.

  \end{enumerate}
\end{protocol}

\begin{itemize}
  \item \textbf{Protocol Commitment Multiplication} (\autoref{prot:commit-mult}) is secure for any $t < n$
  \begin{itemize}
  	\item It assumes that $1/|\mathbb{F}|$ is negligible in the security parameter and if not the step 2 is repeated $u$ times such that $(1/|\mathbb{F})^u$ is negligible.
  \end{itemize}
  \item \textbf{Protocol Perfect Commitment Multiplication} (\autoref{prot:commit-mult-perf}) is secure for any $t < n/3$
  \item The idea in Protocol Commitment Multiplication is that $\P_i$ commits to what he or she claims is the product $ab$ and the proves to each other player $\P_k$ that this was done correctly
  \begin{itemize}
  	\item If $c = ab + \Delta$ for $\Delta \neq 0$ then for each $\P_k$ the proof fails except with probability $1/|\mathbb F|$
  \end{itemize}

  \item The idea in Protocol Perfect Commitment Multiplication is to have $\P_i$ commit to polynomials $f_a,f_b$ and to what it claims is the product $h=f_af_b$ of the polynomials
  \begin{itemize}
  	\item The players check that $h(k) = f_a(k)f_b(k)$ in a number of points sufficient to guarantee that $h=f_af_b$
  \end{itemize}

  \item Let $\pi_{\mathtt{TRANSFER},\mathtt{MULT}}$ be the protocol that
  \begin{itemize}
  	\item Implements the basic commands by simply relaying the commands to $\F_{\mathtt{COM-SIMPLE}}$
  	\item Implements the transfer and multiplication commands by running Protocol Transfer and Protocol Commitment Multiplication on $\F_{\mathtt{COM-SIMPLE}}$
  \end{itemize}

  \item Let $\pi_{\mathtt{PTRANSFER},\mathtt{PMULT}}$ be the protocol that
  \begin{itemize}
  	\item Implements the basic commands by simply relaying the commands to $\F_{\mathtt{COM-SIMPLE}}$
  	\item Implements the transfer and multiplication commands by running Protocol Perfect Transfer and Protocol Perfect Commitment Multiplication on $\F_{\mathtt{COM-SIMPLE}}$
  \end{itemize}
\end{itemize}
\begin{protocol}{Agent $\mathcal S_{\COM}$}{s-com}
  A simulator for $\pi_{\mathtt{PTRANSFER},\mathtt{PMULT}} \diamond \F_{\mathtt{SC}} \diamond \F_{\mathtt{COM-SIMPLE}}$ for the static case
  \begin{itemize}
  	\item \textbf{Basic commands:}
    \begin{itemize}
      \item When any of the basic commands are executed, the simulator will just forward messages back and forth between the environment and $\FCOM$
      \item When a corrupted players commits to a value, $\mathcal{S}_{\COM}$ will store it.
    \end{itemize}
  	\item \textbf{Transfer:} When $\FCOM$ is executing a \texttt{transfer} of some value $a$ from $\P_i$ to $\P_j$, then $\P_i$ has already committed to that values
    \begin{itemize}
    	\item If $\P_i$ is corrupted, the simulator has already stored $a$ and if $\P_j$ is corrupted, then $a$ is leaked from $\FCOM$ as soon as \texttt{transfer} is initiated
      \begin{itemize}
        \item $\mathcal{S}_\COM$ can run its copy of $\pi_{\texttt{PTRANSFER}} \diamond \F_{\texttt{SC}} \diamond \F_{\texttt{COM-SIMPLE}}$ with the environment using as input the correct value $a$ and where it simulates the parts of the honest players. 
      \end{itemize}
      \item If both parties are honest, the transfer protocol is simulated with some random values
      \item If the simulated protocol terminates correctly, the simulator inputs $(\texttt{transfer}, i, cid, j)$ and ask for delivery of outputs on $\COM.\texttt{infl}$.
      \item If $\P_i$ is corrupt and makes some command fail, the simulator will ask for delivery of outputs without inputting $(\texttt{transfer},i,cid,j)$ which will make $\FCOM$ output \texttt{fail}. 
    \end{itemize}
  	\item \textbf{Multiplication:} When $\FCOM$ is executing a \texttt{mult} for $\P_i$ of some values $a,b$, then $\P_i$ has already committed to these values
    \begin{itemize}
    	\item If $\P_i$ is corrupted, the simulated has already stored all values needed and can do $\pi_{\texttt{PMULT}} \diamond \F_{\texttt{SC}} \diamond \F_{\texttt{COM-SIMPLE}}$ with the correct values.
      \item If $\P_i$ is honest $\pi_{\texttt{PMULT}} \diamond \F_{\texttt{SC}} \diamond \F_{\texttt{COM-SIMPLE}}$ is done with random values.
      \item If the simulated protocol terminates correctly, the simulator inputs $(\texttt{mult},i,cid_1,cid_2,cid_3)$ and asks for delivery of outputs on $\COM.\texttt{infl}$.
      \item If $\P_i$ is corrupted and makes some command fail, the simulator will ask for delivery of outputs on $\COM.\infl$ without inputting a command, which will make the $\FCOM$ output fail
    \end{itemize}
  \end{itemize}
\end{protocol}
\begin{itemize}
  \item \textbf{Theorem 5.1} Let $\F_{\mathtt{COM-SIMPLE}}$ be a functionality that has all the commands of $\F_{\mathtt{COM}}$ except the advanced manipulation commands. Then $\pi_{\mathtt{TRANSFER}, \mathtt{MULT}} \diamond \F_{\mathtt{SC}} \diamond \F_{\mathtt{COM-SIMPLE}}$ implements $\FCOM$ in $\Env^{t,sync}$ with statistical security for all $t <n$. Moreover $\pi_{\mathtt{PTRANSFER}, \mathtt{PMULT}} \diamond \F_{\mathtt{SC}} \diamond \F_{\mathtt{COM-SIMPLE}}$ implements $\FCOM$ in $\Env^{t,sync}$ with perfect security for all $t < n/3$.
  \begin{proof} 
    It is proven in the case for perfect security. 
    \begin{itemize}
      \item The simulator $\mathcal{S}_\COM$ is constructed (\autoref{prot:s-com}). The idea is that the simulator will relay messages relayed to the basic commitment commands, while for \texttt{transfer} and \texttt{mult}, it will run internally a copy of $\pi_{\texttt{PTRANSFER},\texttt{PMULT}} \diamond \F_{\texttt{SC}}, \F_{\texttt{COM-SIMPLE}}$ where the instance of $\F_{\texttt{COM-SIMPLE}}$ is used for executing basic commitment commands related to executing $\pi_{\texttt{PTRANSFER}, \texttt{PMULT}}$
      \item The simulator will play the roles of the honest parties 
      \item The messages concerning the basic commands are just relayed and thus for these commands the distribution is identical
      \item During \texttt{transfer}, if either $\P_i$ or $\P_j$ is corrupted the simulator learn the value to be transfer (similarly in \texttt{mult} for $\P_i$). Hence in these cases, the protocol $\pi_{\texttt{PTRANSFER},\texttt{PMULT}}$ is run on the true values
      \item If the players are honest then for both \texttt{transfer} and \texttt{mult}, the environment sees at most $t$ points of polynomials of degree at least $t$ and thus no information on the actual values is given
    \end{itemize}
    What remains to be argued is that the protocol execution is consistent with the inputs-outputs to and from $\FCOM$ i.e. if the protocol finishes successfully, then the output is correct and if the protocol fails this is also consistent with the state of $\FCOM$
    \begin{itemize}
    	\item During \texttt{transfer}
      \begin{itemize}
        \item If $\P_i$ and $\P_j$ are honest, then it is clear that all the proofs will be accepting
        \item If one or both are corrupted and we still reach the point where values are compared, no command failed earlier, At this point there are strictly more than $2t$ honest players and if at most $t$ players said reject, then at least $t+1$ honest players said accept and thus $\P_i$ and $\P_j$ are committed to the same polynomial (which means that they are committed to the same value)
      \end{itemize}
      \item For \texttt{mult} 
      \begin{itemize}
      	\item Even if $\P_i$ is corrupt it must be the case that 
        \[
          f_a(0) = a, f_b(0) = b, h(0) = c \text{ and } \deg(f_a),\deg(f_b) \leq t, \deg(h) \leq 2t
        \]
        \item If the players output \texttt{success} at the end it follows that $f_a(k)f_b(k) = h(k)$ for every player $\P_k$ who remains honest
        \item The assumption that $t < n/3$ implies that there are at least $2t+1$ honest players and thus they determine a degree of at most $2t$ uniquely which means that $f_a(\X)f_b(\X) = h(\X)$ and thus $ab = f_a(0)f_b(0) = h(0) = c$
        \item One can also clearly see that if $\P_i$ remains honest the protocol will always be successful
      \end{itemize}
      \item If the simulated protocol fails at some point because of a corrupted $\P_i$, the simulator outputs \texttt{fail} which is consistent with the internal state of $\FCOM$ since in this case $\mathcal{S}_\COM$ also makes $\FCOM$ fail
    \end{itemize}
  \end{proof}
\end{itemize}

\newpage

%%% Local Variables:
%%% mode: latex
%%% TeX-master: "cpt-noter"
%%% End:
