\section{$\Sigma$ Protocols}
\begin{itemize}
  \item Let $R$ be a binary relation i.e $R \subseteq \{0,1\}^* \times \{0,1\}^*$
  \begin{itemize}
  	\item The only restriction is that if $(x,w) \in R$ then the length of $w$ is at most $p(|x|)$ for some polynomial $p()$
  	\item $w$ is called a \textbf{witness} for $x$
  \end{itemize}
  \item \textbf{$\Sigma$ protocols} are in the following 3 move form where $x$ is a common input to $P,V$ and a $w$ such that $(x,w) \in R$ is a private input to $P$:
  \begin{enumerate}
  	\item $P$ sends a message $a$
  	\item $V$ sends a random $t$ bit string $e$
  	\item $P$ sends a reply $z$, and $V$ decides to accept of reject based on the data he has seen i.e. $x,a,e,z$
  \end{enumerate}
  \item $P$ and $V$ are assumed to be probabilistic polynomial time machines and $P$'s only advantage over $V$ is that he knows $w$ 

  \item \textbf{Definition 1.} A protocol $\mathcal P$ is said to be a $\Sigma$ protocol for a relation $R$ if:
  \begin{itemize}
    \item $P$ is on the 3 move form
    \item \textbf{Completeness:} If $P,V$ follow the protocol on input $x$ and private input $w$ to $P$ where $(x,w) \in R$, the verifier always accepts
  	\item \textbf{The Special Soundness Property:} From any $x$ and any pair of accepting conversations on input $x$, $(a,e,z),(a,e',z')$ where $e \neq e'$, one can efficiently compute $w$ such that $(x,w)$ in $R$
  	\item \textbf{Special Honest-Verifier Zero-Knowledge:} There exists a polynomial-time simulator $M$, which on input $x$ and a random $e$ outputs an accepting conversation of the form $(a,e,z)$ with the same probability distribution as conversations between the honest $P,V$ on input $x$
  \end{itemize}
  \item Define $L_R$ to be the set of $x$'s for which there exists $w$ such that $(x,w) \in L$
  \begin{itemize}
  	\item The special soundness property implies that a $\Sigma$ protocol for $R$ is always an interactive proof system of $L_R$ with error probability $2^{-t}$
  \end{itemize}
  \item \textbf{Lemma 1.} The properties of $\Sigma$ protocols are invariant under parallel composition i.e. repeating a $\Sigma$ protocol for $R$ twice in parallel produces a new $\Sigma$ protocol for $R$ with challenge length $2t$

  \item \textbf{Lemma 2.} If a $\Sigma$ protocol for $R$ exists, then for any $t$, there exists a $\Sigma$ protocol for $R$ with challenge length $t$ 

\end{itemize}

\subsection{Proof of Knowledge}
\begin{itemize}
  \item \textbf{Definition 2.} Let $\kappa()$ be a function from bit strings to the interval $[0..1]$. The protocol $(P,V)$ is said to be a proof of knowledge for the relation $R$ with knowledge error $\kappa$ if the following are satisfied
  \begin{itemize}
  	\item \textbf{Completeness} On common input $x$, if the honest prover $P$ get a private input $w$ such that $(x,w) \in R$ then the verifier always accepts
  	\item \textbf{Knowledge soundness} There exists a probabilistic algorithm $M$ called the knowledge extractor
    \begin{itemize}
  		\item It gets input $x$ and rewindable black-box access to the prover
  		\item It attempts to compute $w$ such that $(x,w) \in R$
  		\item It is required that the following holds:
    \end{itemize}
    \[
      \frac{|x|^c}{\epsilon(x)-\kappa(x)} 
    \]
    where access to $P^*$ counts as one step only and the error $\kappa()$ is the probability that one can convince the verifier without knowing a correct $w$ 
  \end{itemize}

  \item \textbf{Theorem 1.} Let $\mathcal P$ be a $\Sigma$ protocol for relation $R$ with challenge length $t$.
  \begin{itemize}
  	\item $\mathcal P$ is a proof of knowledge with knowledge error $2^{-t}$ 
  \end{itemize}
\end{itemize}

\subsection{OR-Protocols}
\begin{protocol}{$\mathcal P_{OR}$ for a $\Sigma$ protocol $\mathcal P$}{sigma-or}
  \begin{enumerate}
   	\item $P$ computes the first message $a_b$ in $\mathcal P$, using $x_b,w$ as input. \\
  		    P chooses $e_{1-b}$ at random and runs the simulator $M$ on input $x,e_{1-b}$ \\
  		    Let $(a_{1-b},e_{1-b},z_{1-b})$ be the output \\
  		    $P$ sends $a_0, a_1$ to $V$
  	\item $V$ chooses a random $t$ bit string $s$ sends it to $P$
  	\item $P$ sets $e_b = s \oplus e_{1-b}$ and computes the answer $z_b \in \mathcal P$ to challenge $e_b$ using $x_b,a_b,e_b,w$ as input. He sends $e_b, z_0,e_1,z_1$ to $V$
  	\item $V$ checks that $s=e_0 \oplus e_1$ and that conversations $(a_0, e_0, z_0)$, $(a_1, e_1, z_1)$ are accepting conversations in $\mathcal P$, on inputs $x_0$ resp. $x_1$
  \end{enumerate}
\end{protocol}

\begin{itemize}
  \item A basic construction with $\Sigma$ protocols allows a prover to show that given two inputs $x_0$ and $x_1$, he knows $w$ such that either $(x_0,w) \in R$ or $(x_1,w) \in R$, but without revealing which case
  \begin{itemize}
  	\item It is assumed that we are given a $\Sigma$ protocol $\mathcal P$ for $R$
    \begin{itemize}
      \item $x_0$, $x_1$ are assumed to be common input to $P$, $V$ and $w$ is private input to $P$ where $(x_b,w) \in R$ for $b=0$ or $1$
    \end{itemize}
  	\item The idea is to ask the prover to complete two instances of $\mathcal P$ where:
    \begin{itemize}
  		\item For $x_b$ it is done for real
  		\item For $x_{1-b}$ it can be faked using the simulator $M$
  		\item A little freedom is given in choosing the challenges to answers, such that he is able to complete both instances
    \end{itemize}
  \end{itemize}

  \item Let $R_{OR} = \{((x_0,x_1),w) \mid (x_0,w) \in R \text{ or } (x_1,w) \in R\}$. Then we have:
  \begin{itemize}
  	\item \textbf{Theorem 2.} The protocol $\mathcal P_{OR}$ (\autoref{prot:sigma-or}) is a $\Sigma$ protocol for $R_{OR}$. Moreover for any verifier $V^*$, the probability distribution of conversations between $P$ and $V^*$, where $w$ is such that $(x_b, w) \in R$, is independent of $b$
    \begin{proof} 
      It is clear that the protocol has the right 3-move form. \\
      To verify special soundness, let two accepting conversations
      \[
        (a_0,a_1,s,e_0,e_1,z_0,z_1),(a_0,a_1,s',e_0',e_1',z_0',z_1') \text{ with } s \neq s'
      \]
      be given. Then clearly it must be the case that for some $c=0$ or $1$, $e_c \neq e_c'$ and then from $(a_c,e_c,z_c), (a_c,e_c',z_c')$ we can compute $w$ such that $(x_c,w) \in R$, by special soundness of $\mathcal P$. \\
      Honest verifier zero-knowledge is clear given $s$, choose $e_0$, $e_1$ at random subject to $s=e_0  \oplus e_1$ and run $M$ twice, on inputs $(x_0, e_0)$, resp. $(x_1, e_1)$. \\
      Assume the we are given an arbitrary verifier $V^*$. Observer that the distribution of conversations between $P$ and $V^*$ can be specified as follows:
      \begin{itemize}
      	\item They have the form $a_0, a_1, s, e_0, e_1, z_0, z_1$
        \item $a_0$ and $a_1$ are distributed as an honest prover in $\mathcal P$ would choose them because of the honest verifier zero-knowledge of $\mathcal P$
        \item $s$ has whatever distribution $V^*$ outputs, given $x_0,x_1,a_0,a_1$ and $e_0,e_1$ are random subject to $s=e_0 \oplus e_1$
        \item $z_0$ has whatever distribution the honest prover in $\mathcal P$ outputs, given that the input was $x_0$ and the first part of the conversation was $a_0,e_0$ similarly for $z_1$
        \item It is trivial for $z_b$ and it follows from perfect honest verifier zero-knowledge for $z_{1-b}$
        \item Thus from this specification that the distribution does not depend on $b$
      \end{itemize}

    \end{proof}
  \end{itemize}
  \item \autoref{prot:sigma-or} ($\mathcal P_{OR}$) is what is known as \textit{witness indistinguishable} (WI): there are several different values of $w$ that a prover may know that would enable him to complete the protocol successfully. But there is no way one can tell from the conversations which of the possible values he knows.
\end{itemize}

\subsection{Hard Relations}
\begin{itemize}
  \item \textbf{Definition 3.} A relation $R$ is said to be hard if
  \begin{itemize}
  	\item There exists a probabilistic polynomial time algorithm $G$, called the generator, which on input $1^k$ outputs a pair $(x,w) \in R$ where $|x| = k$
  	\item The following holds for all probabilistic polynomial time algorithms:
    \begin{itemize}
  		\item Consider the experiment where we run $G$ on input $1^k$
  		\item Give the $x$ produced to $A$, and let $w_A$ be the output $A$ produces
  		\item Let $p_A(k)$ be the probability that $(x,w_A) \in R$, then $p_{A}(k)$ is negligible in $k$
    \end{itemize}
  \end{itemize}

  \item Let a relation $R$ with generator $G$ be given and assume $R$ has $\Sigma$ protocol $\mathcal P$, then consider the following game played by an arbitrary poly-time verifier $V^*$:
  \begin{enumerate}
  	\item Run $G$ on input $1^k$ to get pairs $(x,w) \in R$. Give $w$ as private input to the prover $P$ in $\mathcal P$
  	\item Let $V^*$ execute $\mathcal P$ with $P$ an arbitrary (polynomial) number of times on common input $x$
  	\item $V^*$ outputs a string $w^*$
  	\item If $(x,w^*) \in R$ we say that $V^*$ wins the game
  \end{enumerate}
  \item \textbf{Definition 4.} $\mathcal P$ is \textbf{witness hiding} if any poly-time $V^*$ wins the above game with only negligible probability

  \item \textbf{Theorem 3.} If $R$ is a hard relation, then the protocol $\mathcal P_{OR}$ for $R_{OR}$ is witness hiding
\end{itemize}

\subsection{Schnorr's protocol}
\begin{protocol}{Schnorr's Protocol}{schnorr-protocol}
  \begin{enumerate}
  	\item $P$ chooses $r$ at random in $Z_q$ and sends $a=g^r \mod p$ to $V$
    \item $V$ chooses a challenge $e$ at random in $Z_{2t}$ and sends it to $P$. Here, $t$ is fixed such that $2^t < q$ 
    \item $P$ sends $z=r+ew \mod q$ to $V$, who checks that $g^z = ah^e \mod p$, that $p,q$ are prime and that $g,h$ have order $q$, and accepts iff this is the case
  \end{enumerate}
\end{protocol}

\begin{protocol}{$\Sigma$ protocol example}{sigma-example}
  Let $p,q$ be chosen as in Schnorr's protocol and let $g_1, g_2, h \in Z^*_p$ be of order $q$. Assume $P$ gets as input $w_1, w_2$ where $h = g^{w_1}g^{w_2} \mod p$. Consider the following protocol:
  \begin{enumerate}
    \item $P$ chooses $r_1,r_2$ at random in $Z_q$ and sends $a = g_1^{r_1} g_2^{r_2} \mod p$ to $V$
    \item $V$ chooses a challenge $e$ at random in $Z_{2^t}$ and sends it to $P$. Where $t$ is fixed such that $2^t < q$ 
    \item $P$ sends $z_1 = r_1 + ew_1 \mod q$, $z_2 = r_2 + ew_2 \mod q$ to $V$, who checks that $g_1^{z_1}g_2^{z_2} = ah^e \mod p$, that $p,q$ are prime and that $g_1,g_2, h$ have order $q$ and accepts iff this is the case
  \end{enumerate}
\end{protocol}

\begin{itemize}
	\item \nameref{prot:schnorr-protocol} (\autoref{prot:schnorr-protocol}) is a $\Sigma$ protocol for the relation $DL$ defined by
  \[
    DL = \{(x,w) \mid x = (p,q,g,h), ord(g) = ord(g) = q, h=g^w\}
  \]
  where $p$ are prime and $q$ are a prime divisor in $p-1$, $g,h \in Z_p^*$ and $w \in Z^q$
  \item \textbf{Lemma.} \autoref{prot:sigma-example} is a $\Sigma$-protocol for the relation
  \[
    \{(x,(w_1,w_2)) \mid x = (p,q,g_1,g_2,h) \text{ and } h = g_1^{w_1} g_2^ {w_2}\}
  \]  
  where is should also be satisfied that $p,q$ are prime, $w_1, w_2 \in Z_q$ and that $g_1, g_2, h \in Z_p^*$ have order $q$ 
  \begin{proof} 
  The protocol is indeed on the 3-move form \\
  Completeness is shown as follows: \\
  Since both $P$ and $V$ follows the protocol it must be the case that $h = g_1^{w_1} g_2^{w_2} \mod p$, that $z_1 =r_1 + ew_1 \mod q$ and $z_2 = r_2 + ew_2 \mod q$ where $r_1,r_2$ are the randomly chosen elements from step 1 and the random challenge $e$ from step 2 and therefore
  \[
    g_1^{z_1}g_2^{z_2} \equiv g_1^{r_1 + ew_1}g_2^{r_2 + ew_2} \equiv g_1^{r_1 + r_2} g^{e(w_1 + w_2)} \equiv a (g^{w_1}g^{w_2})^e \equiv ah^e \cg p
  \]
  This implies completeness since $g_1^{z_1}g_2^{z_2} = ah^e \mod p$ \smallskip \\
  The special soundness property is shown as follows:
  \begin{align*}  
    g_1^{z_1}g_2^{z_2} = ah^e \mod p \land g_1^{z_1'}g_2^{z_2'} = ah^{e'} \mod p &\iff
    g_1^{z_1}g_2^{z_2}h^{-e} = g_1^{z_1'}g_2^{z_2'}h^{-e'} \mod p \\
    &\iff g_1^{z_1-z_1'}g_2^{z_2-z_2'} = h^{e-e'} \mod p \\
    &\iff g_1^{z_1-z_1'}g_2^{z_2-z_2'} = (g_1^{w_1}g_2^{w_2})^{e-e'} \mod p \\
    &\iff (g_1^{z_1-z_1'}g_2^{z_2-z_2'})^{e'-e} = g_1^{w_1}g_2^{w_2} \mod p \\
    &\iff g_1^{(e'-e)(z_1-z_1')}g_2^{(e'-e)(z_2-z_2')} = g_1^{w_1}g_2^{w_2} \mod p \\
  \end{align*}
  This implies the special property since $w_1 = (e'-e)(z_1 - z_1') \mod q$ and $w_2 = (e'-e)(z_1 - z_1') \mod q$ \smallskip \\
  The protocol is shown to be honest verifier zero-knowledge as follows: \\
  To simulate a conversation simply choose $e \in_R 2^{t}$, $z_1 \in_R Z_q$ and $z_2 \in_R Z_q$ and compute $a = g_1^{z_1}g_2^{z_2}h^{-e}$. The conversation $(a, e, (z_1,z_2))$ has the same distribution as a conversation between an honest prover and a honest verifier. This follows from it being an accepting conversation since
  \[
    ah^e = g_1^{z_1}g_2^{z_2}h^{-e}h^e = g_1^{z_1} g_2^{z_2}
  \]
  $z_1$ and $z_2$ clearly has the same distribution since they are just randomly chosen elements and since $a$ is based on randomly chosen elements as in the real case. It must be the case that $a$s distribution is the same as in the real case. \smallskip \\
  \end{proof}
\end{itemize}

\newpage

%%% Local Variables:
%%% mode: latex
%%% TeX-master: "cpt-noter"
%%% End:
