\section{The UC model}
\subsection{Ideal Functionalities}
\begin{itemize}
  \item An ideal functionality will be an interactive agent
  \item Each ideal functionality has a name $\F$
  \item The interface of $\F$ is defined as follows
  \begin{itemize}
  	\item It has $n$ inports named $\F.\mathsf{in}_1, \dots, \F.\mathsf{in}_n$ and $n$ outports named $\F.\mathsf{out}_1, \dots, \F.\mathsf{out}_n$
  	\item The $2n$ ports are called the protocol ports
  	\item In addition to the protocol ports $\F$ has two special ports
    \begin{itemize}
  		\item An inport $\F.\mathsf{infl}$ called the influence port
  		\item An outport $\F.\mathsf{leak}$ called the leakage port
    \end{itemize}
  \end{itemize}
\end{itemize}

\subsection{Protocols}
\begin{itemize}
  \item A simple protocol $\pi$ consists of just $n$ parties $\P_1, \dots, \P_n$, where each $\P_i$ is an agent
  \begin{itemize}
  	\item The protocol is the interactive system $\pi = \{\P_1, \dots, \P_n\}$
  	\item A simple protocol has a protocol name $\F$
    \begin{itemize}
  		\item The protocol $\pi$ and the ideal functionality $\F$ that $\pi$ is supposed to implement has the same name
    \end{itemize}
  	\item It also has a resource name $R$ which is the name of the resource that $\pi$ uses for communication
  	\item The agent $\P_i$ is called a simply party and it has six ports 
    \begin{itemize}
  		\item The port structure of $\P_i$ is derived from the names $\F$ and $R$
  		\item It has an inport $\F.\mathtt{in}_i$ and an outport $\mathsf{F}.\mathtt{out}_i$ which is called \textbf{protocol ports}

  		\item It has an outport named $R.\mathsf{in}_i$ and an inport named $R.\mathsf{out}_i$ which is called \textbf{resource ports}
  		\item It has an inport named $R.\mathsf{infl}_i$ and an inport named $R.\mathsf{leak}_i$ which is called the special ports and are used to model corruption of $\P_i$ and to clock $\P_i$
    \end{itemize}
  \end{itemize} 

  \item All parties have the following standard corruption behavior:
  \begin{itemize}
  	\item If a party $\P_i$ receives a special symbol (passive corrupt) on $R.\mathtt{infl}_i$, then $\P_i$ returns its internal state $\sigma$ on $R.\mathtt{leak}_i$
    \begin{itemize}
  		\item The internal state $\sigma$ consists of all randomness used by the party so far along with all inputs sent and received on its ports and the messages in the message queues of its inports
    \end{itemize}
  	\item If a party $\P_i$ receives a special symbol (active corrupt) on $R.\mathtt{infl}_i$ then $\P_i$ outputs its current state on $R.\mathtt{leak}_i$ and starts executing the following rules and only these rules:
    \begin{itemize}
  		\item On input $(\mathtt{read}, \P)$ on $\mathsf{R}.\mathtt{infl}_i$, where $\P$ is an inport of $\P_i$, it reads the next message $m$ on $\P$ and returns $m$ on $R.\mathtt{leak}_i$
  		\item On input $(\mathtt{send}, \P, m)$ on $\mathsf{R}.\mathtt{infl}_i$, where $\P$ is an outport of $\P_i$, it sends $m$ to $\P$
    \end{itemize}
  \end{itemize}
\end{itemize}

\subsection{UC-Theorem}
\begin{itemize}
  \item Two systems are indistinguishable to an environment $\mathcal Z$ if $\mathcal Z$ cannot tell them apart (except with negligible advantage) by sending and receiving messages on the open ports of the systems
  \begin{itemize}
  	\item Two systems $S$, $F$ are called \textbf{indistinguishable} to a class $Z$ of environments if they are indistinguishable to all $\mathcal Z \in Z$ written $S \stackrel{Z}{\equiv} F$
  \end{itemize}
  \item \textbf{Definition 4.10 (Security for Simple Protocols)} Let $ST$ and $AT$ denote arbitrary protocol names. Let $F_{ST}$ be any ideal functionality with name $ST$, let $\pi_{st}$ be any simple protocol with protocol name $ST$ and resource name $AT$ and let $F_{AT}$ be any ideal functionality with name $AT$. Then $\pi_{ST} \diamond F_{AT}$ securely implements $F_{ST}$ in environments from $Z$ if there exists a simulator $\mathcal{S}$ for $\pi_{ST}$ such that $\pi_{ST} \diamond F_{AT} \stackrel{Z}{\equiv} F_{ST} \diamond \mathcal S$.
  \begin{itemize}
  	\item It is also written as $\pi_{ST} \diamond F_{AT} \stackrel{Z}{\geq} F_{ST}$
  	\item $\pi_{ST} \diamond F_{AT}$ is at least as secure as $F_{ST}$ in environments from $Z$
  \end{itemize}
  \item It is required that if one takes a protocol $\pi$ and an environment $\mathcal Z$ for $\pi$, then $\mathcal Z \diamond \pi$ is again an environment
  \item \textbf{Definition 4.13} Let $\text{Pro}$ be the set of simple and composed protocols in which all simple parties follows the rules for clocked entities and are recursive polytime
  \item \textbf{Lemma 4.14} If $\pi_F \in \text{Pro}$ is a protocol with protocol name $\F$ and resource name $R$ and $\pi_R \in \text{Pro}$ is a protocol with protocol name $R$ and $\pi_{\F} \diamond \pi_R \neq \bot$ then $\pi_{\F} \diamond \pi_R \in \text{Pro}$
  \item \textbf{Definition 4.15} Let $\text{Sim}$ be the set of interactive systems $\mathcal S$ that are a simulator for some protocol. i.e. for $\mathcal S \in \text{Sim}$, it holds that
  \begin{itemize}
  	\item There exists a protocol $\pi_{\F}$ (composed or simple) with protocol name $\F$ and an ideal functionality $\F_{F}$ with name $F$ such that $\pi_{F}$ and $\F_{F} \diamond \mathcal S$ have the same special ports
  	\item $\mathcal S$ follows the rules for clocked entities and is recursive polytime
  	\item $\mathcal S$ is corruption preserving
  	\item $\mathcal S$ is clock preserving
  \end{itemize}
  \item \textbf{Lemma 4.16} If $\mathcal S \in \text{Sim}$ and $\mathcal T \in \text{Sim}$ and $\mathcal T$ have two open special ports that connect it to the ideal functionality ports of $\mathcal S$ and $\mathcal S \diamond \mathcal \tau \neq \bot$, then $\mathcal S \diamond \mathcal T \in \text{Sim}$
  \item \textbf{Definition 4.17} $\text{Env}$ is an environment class if the following holds:
  \begin{itemize}
  	\item Each $\mathcal Z \in \text{Env}$ has the open port structure of an environment for some simple or composed protocol
  	\item For all $\mathcal Z \in \text{Env}$ and all $\pi \in \text{Pro}$, where $\pi \diamond \mathcal Z \neq \bot$ and the protocol ports of $\pi$ connect to the protocol ports of $\mathcal Z$ in $\pi \diamond \mathcal Z$, it holds that $\pi \diamond \mathcal Z \in \text{Env}$.
  	\item For all $\mathcal Z \in \text{Env}$ and all $\mathcal S \in \text{Sim}$, where $\mathcal S \diamond \mathcal Z \neq \bot$ and the simulation ports of $\mathcal S$ connect to the special ports of $\mathcal Z$, it holds that $\mathcal S \diamond \mathcal Z \in \text{Env}$
  \end{itemize}
  \item An environment $\mathcal Z$ is called \textbf{recursive polytime} if it is polytime and it makes at most an expected polynomial number of calls before it makes its guess
  \item \textbf{Proposition 4.18} Let $\text{Env}^{poly}$ be the set of all recursive polytime systems that have an open port structure of an environment for some simple or composed protocol. Then $\text{Env}^{poly}$ is an environment class
  \item \textbf{Definition 4.19 (UC Security for Protocols)} Let $\F_F$ be an ideal functionality with name $F$, let $\pi_F$ be a protocol with protocol name $\F$ and resource name $R$, and let $\F_R$ be an ideal functionality with name $R$. Let $\text{Env}$ be an environment class.
  \begin{itemize}
  	\item Then $\pi_F \diamond \F_R$ securely implements $\F_F$ in environments $\text{Env}$ if there exists a simulator $\mathcal S \in \text{Sim}$ for $\pi_F$ such that $\pi_{F} \diamond \F_R \stackrel{\text{Env}}{\equiv} \F_F \diamond \mathcal S$
  	\item It is written $\pi_F \diamond \F_R \stackrel{\text{Env}}{\geq} \F_F$
  	\item $\pi_F \diamond \F_R$ is at least as secure as $\F_F$ for environments in $\text{Z}$
  \end{itemize}
  \item \textbf{Theorem 4.20 (The UC Theorem)} Let $\text{Env}$ be an environments class. Let $\pi_F \in \text{Pro}$ be a protocol with protocol name $F$ and resource name $G$. Let $\pi_G$ be a protocol with protocol name $G$ and resource name $H$ for which $\pi_F \diamond \pi_G \neq \bot$. Let $\F_F$, $\F_G$ and $\F_H$ be ideal functionalities with names $\F$, $\mathsf{G}$ and $\mathsf{H}$. If $\pi_{\F} \diamond F_G \stackrel{\text{Env}}{\geq} \F_{F}$ and $\pi_{\G} \diamond \F_H \stackrel{\text{Env}}{\geq} \F_{G}$, then $(\pi_F \diamond \pi_G) \diamond \F_H \stackrel{\text{Env}}{\geq} \F_F$
  \begin{proof}
    First the premises of the theorem is written out to make the easier to use. \\
    It is assumed that $\pi_\F \diamond \sgeq F_\G$ thus there exists a simulator $\mathcal S \in \Sim$ for $\pi_\F$ such that
    \begin{equation}
      \pi_\F \diamond \F_G \diamond \mathcal Z_1 \stackrel{\text{stat}}{\equiv} \F_F \diamond \mathcal S \diamond \mathcal Z_1 \label{eq:uc-assum1}
    \end{equation}
    for all $\mathcal Z_1 \in \Env$ for which the compositions make sense. \smallskip \\
    It is also assumed that $\pi_{G} \diamond F_H \sgeq F_G$ thus there exists a simulator $\mathcal T \in \Sim$ for $\pi_G$ such that
    \begin{equation}  
      \pi_G \diamond \F_H \diamond \mathcal Z_2 \stackrel{\text{stat}}{\equiv} \F_G \diamond \mathcal T \diamond \mathcal Z_2 \label{eq:uc-assum2}
    \end{equation}
    for all $\mathcal Z_2 \in \Env$ for which the compositions make sense. \smallskip \\
    Since we want to prove that $\pi_F \diamond \pi_G \diamond \F_H \sgeq \F_F$ this means that there should exist a simulator $\mathcal U \in \Sim$ for $\pi_F \diamond \pi_G$ such that
    \begin{equation}  
      \pi_F \diamond \pi_G \diamond \F_H \diamond \mathcal Z \stackrel{\text{stat}}{\equiv} \F_F \diamond \mathcal U \diamond \mathcal Z \label{eq:uc-conclusion}
    \end{equation}
    for all $\mathcal Z \in \Env$ for which the compositions make sense. \bigskip \\
    Then it is shown that
    \begin{equation}
      \pi_F \diamond \pi_G \diamond \F_H \diamond \mathcal Z \stackrel{\text{stat}}{\equiv} \pi_F \diamond \F_G \diamond \mathcal T \diamond \mathcal Z \label{eq:uc-subresult1}
    \end{equation}
    by letting $\mathcal Z_2 \stackrel{\text{def}}{=} \pi_F \diamond \mathcal Z$ then we just have to prove that
    \begin{equation}
      \pi_G \diamond \F_H \diamond \mathcal Z_2 \stackrel{\text{stat}}{\equiv} \F_G \diamond \mathcal T \diamond \mathcal Z_2 
    \end{equation}
    which follows directly from Eq. (\ref{eq:uc-assum2}) \smallskip \\
    Then it is shown that
    \begin{equation}
      \pi_F \diamond \F_G \diamond \mathcal T \diamond \mathcal Z \stackrel{\text{stat}}{\equiv} \F_F \diamond \mathcal S \diamond \mathcal T \diamond \mathcal Z \label{eq:uc-subresult2}
    \end{equation}
    by letting $\mathcal Z_1 \stackrel{\text{def}}{=} \mathcal T \diamond \mathcal Z$ then we just have to prove that
    \begin{equation}
      \pi_F \diamond \F_G \diamond \mathcal Z_1 \stackrel{\text{stat}}{\equiv} \F_F \mathcal S \diamond Z_1
    \end{equation}
    which follows directly from Eq. (\ref{eq:uc-assum1}) and thus per Eq. (\ref{eq:uc-subresult1}) and (\ref{eq:uc-subresult2}) and the transitivity of $\stackrel{\text{stat}}{\equiv}$ one can conclude that
    \begin{equation}  
      \pi_F \diamond \pi_G \diamond \F_H \diamond \mathcal Z \stackrel{\text{stat}}{\equiv} \F_F \diamond \mathcal S \diamond \mathcal T \diamond \mathcal Z
    \end{equation}
    which if $\mathcal U \stackrel{\text{def}}{=} \mathcal S \diamond \mathcal T$, then $\mathcal U \in \Sim$ and if plugging this in we get exactly Eq. (\ref{eq:uc-conclusion}) which was what we wanted to prove
 \smallskip \\

  \end{proof}
\end{itemize}


\newpage

%%% Local Variables:
%%% mode: latex
%%% TeX-master: "cpt-noter"
%%% End:
