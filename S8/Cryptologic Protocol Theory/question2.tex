\section{Zero-Knowledge}
\begin{itemize}
  \item Loose definition of \textbf{zero-knowledge}: a protocol is zero-knowledge if it communicates exactly the knowledge that was intended, an no (zero) extra knowledge
  \item The user, who is the one wanting to convince the other about the truth of some claim will be called the \textbf{Prover} $P$
  \item The host, who is interested in checking that the claim is true will be called the \textbf{verifier} $V$
\end{itemize}

\subsection{Simple Example}
\begin{protocol}{Simple Protocol}{simple-zero-knowledge}
  \begin{enumerate}
  	\item If the prover claims to be $A$ the verifier chooses a random message $M$ and sends the ciphertext $C = P_A(M)$ to the prover
  	\item The prover decrypts $C$ using $S_A$ and sends the result $M'$ to the verifier
  	\item The verifier accepts the identity of the prover if and only if $M' = M$
  \end{enumerate}
\end{protocol}
\begin{itemize}
  \item \autoref{prot:simple-zero-knowledge} is not zero knowledge, since an adversary could pretend to be the verifier and forward the $C$ message from the real verifier thereby using the Prove to falsely prove that it itself was the Prover
  \begin{itemize}
  	\item Thereby an adversary can use the information send which is not optimal
    \item The verifier needs to somehow show that he knows what the message decrypts to without revealing it before the prover reveals whether he can decrypt it
  \end{itemize}

  \item Assume that we have a commitment scheme that lets the prover commit to any message that can be encrypted by the public key system
  \item Let $\text{commit}_{pk}(r,M)$ denote a commitment to message $M$
  \item \autoref{prot:advance-zero-knowledge} is a way to fix \autoref{prot:simple-zero-knowledge} such that it becomes zero knowledge using commitment schemes

\end{itemize}
\begin{protocol}{Zero Knowledge Version of Simple Protocol}{advance-zero-knowledge}
  \begin{enumerate}
  	\item If the prover claims to be $A$, the verifier chooses a random messages $M$, and sends the cipher text $C=P_A(M)$ to the prover
  	\item The prover decrypts $C$ using $S_A$ and sends a commitment to the result $\text{commit}_{pk}(r,M')$ to the verifier
  	\item The verifier sends $M$ to the prover
  	\item The prover checks if $M = M'$. If not he stops the protocol he opens the commitments, i.e. he sends $r,M'$ to the verifier
  	\item The verifier accepts the identity of the prover if and only if $M' = M$ and the pair $r,M'$ correctly opens the commitment 
  \end{enumerate}
\end{protocol}

\subsection{Interactive Proof Systems}
\begin{itemize}
  \item The protocols defined will take place as interactions between two \textbf{Interactive Turing Machines} i.e. ordinary probabilistic Turing Machines that are in addition equipped with communication tapes allowing a machine to send and receive messages from the other one
  \item For an interactive proof system the following is assumed
  \begin{itemize}
  	\item The prover $P$ has infinite computing power
  	\item The verifier $V$ is polynomial time bounded
  \end{itemize}
  \item The machines get a common input string $x$
  \begin{itemize}
  	\item Running the machines an input $x$ results in $V$ outputting \textbf{accept} or \textbf{reject}
  	\item And thus the pair $(P,V)$ accepts or rejects $x$ accordingly
  \end{itemize}
  \item A binary language $L \subset \{0,1\}^*$ is given and is specialised to the concrete statement that a certain logical statement is true namely that $x \in L$
  \item \textbf{Definition 4} The pair $(P,V)$ is an interactive proof system for $L$ if it satisfies the following two conditions
  \begin{itemize}
  	\item \textbf{Completeness:} If $x \in L$, then the probability that $(P,V)$ rejects $x$ is negligible in the length of $x$
  	\item \textbf{Soundness:} If $x \notin L$, then for any prover $P^*$, the probability that $(P^*,V)$ accepts $x$ is negligible in the length of $x$
  \end{itemize}
  \item There is no way to cheat the verifier even using infinite computing power
\end{itemize}

\subsection{Interactive Arguments}
\begin{itemize}
  \item Another variant of Interactive proof systems is know as \textbf{Interactive Arguments} and has more direct relations to practical protocols
  \begin{itemize}
  	\item In this type of protocol we want the prover to be polynomial time
  	\item We are only concerned about the polynomial time provers cheating the verifier
  \end{itemize}

  \item The simplest way to define an interactive argument for a language $L$, is to say that it is an interactive proof system, but with two changes
  \begin{itemize}
  	\item The honest prover is required to be probabilistic polynomial time
    \begin{itemize}
  		\item The only advantage over the verifier is that it has a private auxiliary input
    \end{itemize}
  	\item The completeness condition says that for every $x \in L$, there is an auxiliary input that allows the prover to convince the verifier almost always
  	\item The soundness condition says "for any probabilistic polynomial time prover", instead of "for any prover"
  \end{itemize}
\end{itemize}

\subsection{Zero-Knowledge}
\begin{itemize}
% - Zero-Knowledge can be seen as an extra property that an interactive proof system might have
% - We want to ensure that the whatever strategy the verifier follows and whatever priori knowledge he may have, he learns nothing new except for the truth of the provers claim
% 	- Done by requiring assuming the prover's claim is true the interaction between the prover and verifier can be efficiently simulating without interacting with the prover
  \item \textbf{The simulation paradigm:} by seeing $Y$ a party learns nothing more than $X$ if $Y$ can be efficiently generated from $X$
  \item A verifier that tries to cheat the prover can be modelled by an arbitrary probabilistic polynomial time machine $V^*$
  \begin{itemize}
  	\item It gets an auxiliary input $\delta$ of length at most some fixed polynomial of the length of input $x$
  	\item The auxiliary input represents the prior knowledge that $V^*$ might have
  \end{itemize}

  \item A conversation between $P$ and any verifier $V^*$ is the ordered concatenation of all messages sent in an execution of the protocol
  \begin{itemize}
  	\item This means that the pair $(P,V^*)$ can be modelled as a machine that gets input $x$ and $\delta$ (only to $V^*$) and outputs a conversation
  \end{itemize}
		
  \item \textbf{Definition 5} An interactive proof system of argument $(P,V)$ for a language $L$ is zero-knowledge if for every probabilistic polynomial time verifier $V^*$, there is a simulator $M_{V^*}$ running in expected polynomial time, such that we have $M_{V^*} \sim^c (P,V)$ on input $x \in L$ and arbitrary $\delta$ (as input to $V^*$ only)
    \begin{itemize}
      \item If for some protocol $M_{V^*} \sim^p (P,V)$ or $M_{V^*} \sim^s (P,V)$ then the protocol is respectively \textbf{perfect zero-knowledge} and \textbf{statistical zero-knowledge}
  \end{itemize}
  \item The simulator has some degrees of freedom that a prover does not have when executing the real protocol and thus it does it does not violate the general definition it can e.g. generate messages in any order it wants
\end{itemize}

\subsection{Zero-Knowledge Protocol Example}
\begin{protocol}{Graph Isomorphism Zero-Knowledge Protocol}{graph-isomorph-zero}
  The protocol works by repeating sequentially the following steps $n$ times:
  \begin{enumerate}
  	\item $P$ chooses a random permutation $\phi$ on $n$ points and sends $H=\phi(G_0)$ to $V$
    \item $V$ chooses at random a bit $b$, and sends it to $P$.
    \item If $b=0$, $P$ sets $\psi = \phi^{-1}$ else he sets $\psi = \pi\phi^{-1}$. He sends $\psi$ to $V$ who checks that $\psi(H) = G_b$ and rejects immediately if not
  \end{enumerate}
  The verifier $V$ only accepts if all $n$ iterations were completed successfully
\end{protocol}
\begin{itemize}
  \item \autoref{prot:graph-isomorph-zero} is a protocol for the graph isomorphism problem: 
  \begin{itemize}
  	\item The common input is a pair of graphs $G_0$, $G_1$ with $n$ nodes
    \item The prover claims the graphs is isomorphic i.e. there exists a permutation $\pi$ such that permuting the nodes of $G_0$ according to $\pi$ one obtains the graph $G_1$ 
    \[
      \pi(G_0) = G_1
    \]
    \item $n$ is used as a measure of the length of the input
    \item The prover does not need to be infinitely powerful, put it is allowed by the definition of proof systems, it is enough that he knows an isomorphism $\pi$ 
  \end{itemize}
  \item No general probabilistic poly-time algorithm is known for deciding if two graphs are isomorphic
  \item \textbf{Theorem} \autoref{prot:graph-isomorph-zero} is a Zero-Knowledge protocol for the Graph Isomorphism Problem
  \begin{proof} 
    It is first shown that it is a proof system \smallskip \\
    Completeness is trivial. If $\pi(G_0) = G_1$ and $\psi(G_0) = H$, then it trivially follows that $V$'s check will be satisfied for both values of $b$ \\
    Soundness can be argued as follows: 
    \begin{itemize}
    	\item We should show that we are able to prove something assuming that the prover's claim is wrong i.e. $G_0$ is not isomorphic to $G_1$ in this case
      \item Assume that in one of the $n$ iterations $P$ can answer both values of $b$ with a permutation that satisfies $V$'s check
      \item Let $\psi_0$, $\psi_1$ be permutations send in reponse to $b=0,1$ 
      \item Since $V$'s checks are satisfies it must be the case that $\psi_0(H) = G_0$ and $\psi_1(H) = G$
      \item This means that $G_0$ is isomorphic to $G_1$ under the isomorphism $\psi_1 \psi_0^{-1}$ which is a clear contradiction
      \item It must thus be the case that in all $n$ iterations, the prover is able to answer at most one of the $2$ possible values of $b$
      \item The probability of acceptance is at most $2^{-n}$ which is negligible in $n$
    \end{itemize}
    The protocol is shown to be perfect zero-knowledge using the following algorithm for a simulator $M$:
    \begin{enumerate}
    	\item Start the machine $V^*$ i.e. give it inputs $G_0, G_1$ and possibly some auxiliary input $\delta$ and random input bits for $V^*$
      \item To simulate one iteration, execute the following loop:
      \begin{enumerate}
      	\item Choose a bit $c$ and a permutation $\psi$ at random. Set $H=\psi^{-1}(G_c)$ and send $H$ to $V^*$ 
        \item Get $b$ from $V^*$. If $b=c$, output $H,b,\psi$ and exit the loop. Else, reset $V^*$ to its state just before the last $H$ was chosen, and go to step 2a
      \end{enumerate}
      If the simulation of all $n$ iterations have been completed stop, else start at Step 2a again
    \end{enumerate}
    To show that this simulator works two things needs to be shown: $M$ runs in expected polynomial time, and the distribution output by $M$ is exactly the same as in the real protocol run. \\
    It is first shown that $M$ runs in expected polynomial time:
    \begin{itemize}
    	\item By definition of zero-knowledge the correctness of a simulation is proved assuming that $P$'s claim is true here it means that $G_0$ is isomorphic to $G_1$
      \item Let $S$ be the set of all graphs isomorphic to $G_0$ (or $G_1$)
      \item The distribution generated in the simulation is thus the same as in the real protocol (uniform of $S$) and is in particular independent of $c$
      \item Thus it follows that the $b$ chosen by $V^*$ must also be independent of $c$ i.e. $\text{P}(c=b) = \frac12$
      \item Thus the expected number of times a loop is done to simulate one iteration is $2$ and thus the whole simulation takes expected $2n$ times the time to go through the loop once which is polynomial in $n$
    \end{itemize}
    Then the output distribution is shown to be the same as in the real case  
    \begin{itemize}
    	\item The simulator produces in every iteration a triple $(H,c,\psi)$, and the $V^*$ outputs $b$ having seen $H$
      \item $H$ has the same distribution as in the real case as it is uniform over $S$ 
      \item It follows that $b$ sent by $V^*$ is also distributed as in the real conversation since $V^*$ chooses $b$ based only on its input and $H$
      \item Since $H$ and $c$ the choice of whether to keep $H$ is does not depend on the choice of $H$ 
      \item Therefore the pair $(H,b)$ are distributed as in the real protocol
      \item $\psi$ is just a random permutation mapping $H$ to $G_b$ as in the real case
      \item Thus the distribution of $M$ matches the real protocol exactly
    \end{itemize}
  \end{proof}
\end{itemize}

\subsection{Rewinding lemma}
\begin{itemize}
  \item If a proof system $(P,V)$ for language $L$ is given and there exists a probabilistic poly-time machine $M_{L}$ with the property that $(P,V) \sim^p M$ on input $x \in L$.then $M$ is called a perfect \textbf{honest-verifier simulator}
  \item \textbf{Lemma 3.1} The rewinding lemma: Let $(P,V)$ be a proof system for language $L$, and let $M$ be a perfect honest-verifier simulator for $(P,V)$. Assume that conversation have the form $(a,b,z)$, where $P$ sends $a$, $V$ responds with a random bit $b$ and $P$ replies with $z$. Then $(P,V)$ is perfect zero-knowledge	
  \begin{proof} 
    Since there exists a honest-verifier simulator $M$ it is the case that $(P,V) \sim^p M$ on input $x \in L$ and honest verifier $V$ and thus it must be the case that the distribution of $(a,b,z)$ is the same for the proof system $(P,V)$ and for the simulator $M$. \smallskip \\
    To prove the lema a new simulator $M_{V^*}$ is created where it is the case that $(P,V^*) \sim^p M_{V^*}$ for any verifier $V^*$ as follows
    \begin{enumerate}
      \item Initialize $M$ and $V^*$
      \item To simulate one iteration the following loop is executed 
      \begin{enumerate}
        \item Simulate an iteration using $M$ as $(a,b,z)$ 
        \item Send $a$ to $V^*$
        \item Get $b'$ from $V^*$ if $b=b'$ then output $(a,b,z)$ else reset $V^*$ and go to step 2(a)
      \end{enumerate}
    \end{enumerate}
    To show that this is zero knowledge we first show that the simulator runs in expected polynomial time: 
    \begin{itemize}
    	\item Since $b$ is chosen at random and is clearly independent of $b'$, the probability that $b=b'$ is $1/2$
       \item The loop terminates with probability $1/2$ and therefore the expected number of times the loop runs is $2$. Which implies that the simulator uses expected $2n$ steps which is expected polynomial time.
    \end{itemize}
    Then perfect zero-knowledge is proven i.e. the output distributions are the same
    \begin{itemize}
      \item $a$ is distributed as in the real case since it does not depend $b$
      \item The distribution of $b$ must be the same as in the real case, since this only depends on $a$ and the verifies arbitrary input
      \item $z$ only depends on $a$ and $b$ to be a correct challenge it follows that $z$ is also distributed as in the real case. 
    \end{itemize}
  \end{proof}
\end{itemize}

\newpage

%%% Local Variables:
%%% mode: latex
%%% TeX-master: "cpt-noter"
%%% End:
